%%%%%%%%%%%%%%%%%%%%%%%%%%%%%%%%% LAB-5 %%%%%%%%%%%%%%%%%%%%%%%%%%%%%%%%%%
%>>>>>>>>>>>>>>>>>>>>>>>>>> ПЕРЕМЕННЫЕ >>>>>>>>>>>>>>>>>>>>>>>>>>>>>>>>>>>
%>>>>> Информация о кафедре
%\newcommand{\year}{2021 г.}  % Год устанавливается автоматически
\newcommand{\city}{Санкт-Петербург}  %  Футер, нижний колонтитул на титульном листе
\newcommand{\university}{Национальный исследовательский университет ИТМО}  % первая строка
\newcommand{\department}{Факультет программной инженерии и компьютерной техники}  % Вторая строка
\newcommand{\major}{Направление системного и прикладного программного обеспечения}  % Треьтя строка
%<<<<< Информация о кафедре

%>>>>> Назание работы
\newcommand{\lab}{Лабораторная работа}
\newcommand{\labnumber}{№ 2}                    % порядковый номер работы
\newcommand{\subject}{Основы профессиональной деятельности}         % учебный предмет
\newcommand{\labtheme}{Исследование работы БЭВМ}            % Тема лабораторной работы
\newcommand{\variant}{№ 1033}                % номер варианта работы

\newcommand{\student}{Тюрин Иван Николаевич}    % определение ФИО студента
\newcommand{\studygroup}{P3110}                 % определение учебной группы 
\newcommand{\teacher}{Клименков С. В.,\\[1mm]     % ФИО лектора
                        Ларочкин Г. И.}          % ФИО практика
%<<<<<<<<<<<<<<<<<<<<<<<<<< ПЕРЕМЕННЫЕ <<<<<<<<<<<<<<<<<<<<<<<<<<<<<<<<<<<


%>>>>>>>>>>>>>>>>>>>>>> ПРЕАМБУЛА >>>>>>>>>>>>>>>>>>>>>>>>>
\include{preamble}

\newcommand{\Rarr}{\Rightarrow}
\newcommand{\LRarr}{\Leftrightarrow}
\newcommand{\rarr}{\xrightarrow{}}
\newcommand{\rt}{\right}
\newcommand{\lt}{\left}

\makeatletter
\newcommand{\xLeftrightarrow}[2][]{\ext@arrow 0359\Leftrightarrowfill@{#1}{#2}}
\makeatother

\makeatletter
\newcommand{\xRightarrow}[2][]{\ext@arrow 0359\Rightarrowfill@{#1}{#2}}
\makeatother
%<<<<<<<<<<<<<<<<<<<<<< ПРЕАМБУЛА <<<<<<<<<<<<<<<<<<<<<<<<<



%%%%%%%%%%%%%%%%%%% СОДЕРЖИМОЕ ОТЧЕТА %%%%%%%%%%%%%%%%%%%%%
%>>>>>>>>>>>>>>> ''''''''''''''''''''''' >>>>>>>>>>>>>>>>>>
\begin{document}


%>>>>>>>>>>>>>>>> ОПРЕДЕЛЕНИЕ НАЗВАНИЙ >>>>>>>>>>>>>>>>>>>>
% Переоформление некоторых стандартных названий
%\renewcommand{\chaptername}{Лабораторная работа}
\renewcommand{\chaptername}{\lab\ \labnumber} % переименование глав
\def\contentsname{Содержание} % переименование оглавления
%<<<<<<<<<<<<<<<< ОПРЕДЕЛЕНИЕ НАЗВАНИЙ <<<<<<<<<<<<<<<<<<<<


%>>>>>>>>>>>>>>>>> ТИТУЛЬНАЯ СТРАНИЦА >>>>>>>>>>>>>>>>>>>>>
\include{titlepage}
%<<<<<<<<<<<<<<<<< ТИТУЛЬНАЯ СТРАНИЦА <<<<<<<<<<<<<<<<<<<<<


%>>>>>>>>>>>>>>>>>>>>> СОДЕРЖАНИЕ >>>>>>>>>>>>>>>>>>>>>>>>>
% Содержание
\tableofcontents
%<<<<<<<<<<<<<<<<<<<<< СОДЕРЖАНИЕ <<<<<<<<<<<<<<<<<<<<<<<<<


%%%%%%%%%%%%%%%%%%%%%%% КОД РАБОТЫ %%%%%%%%%%%%%%%%%%%%%%%%
%>>>>>>>>>>>>>>>>>>>'''''''''''''''''>>>>>>>>>>>>>>>>>>>>>
\newpage
\Chapter{\lab\ \labnumber}{\labtheme}{}
\Section{Задание варианта \variant}
\begin{center}
    , , ,
\end{center}

\indent
По выданному преподавателем варианту определить функцию, вычисляемую программой, область представления и область допустимых значений исходных данных и результата, выполнить трассировку программы, предложить вариант с меньшим числом команд. При выполнении работы представлять результат и все операнды арифметических операций знаковыми числами, а логических операций набором из шестнадцати логических значений.\\

\begin{center}
    \includegraphics[width=0.15\textwidth]{task}
\end{center}

\begin{center}
    ' ' ' 
\end{center}

\Section{Описание программы}

\Subsection{Назначение программы и реализуемые ею функция (формула)}

Описание программы представлено в таблице \ref{tab:program}.\\
Смысловую часть программы можно изобразить формулой:\\ $R=A\And(C-B)$. Программа считает разность переменных $C$ и $B$, затем логически умножает полученное значение на $A$.

\begin{table}[htbp]
    \renewcommand{\arraystretch}{1.2} % Установка высоты строки таблицы
    \centering
    \resizebox{\columnwidth}{!}{%
    \begin{tabular}{|l|l|l|l|}
            \hline
           Адрес, (0x)   & Данные/      & Мнемоника & Результат\\
                         & /Команда, (0x)&           &           \\\hline
           075 &    2075 & --           & Данные (A)\\\hline
           076 &    A082 & --           & Данные (B)\\\hline
           077 &    A082 & LD 0x082     & AC = C \\\hline
           078 &    6076 & SUB 0x076    & AC = C -- B \\\hline
           079 &    E080 & ST 0x080     & Сохранение промеж. рез.:\\
               &         &              & D = C -- B\\\hline
           07A &    0200 & CLA          & AC = 0x0000 = 0\\\hline
           07B &    0280 & NOT          & AC = \textasciicircum AC = 0xFFFF = --1 \\\hline
           07C &    2075 & AND 0x075    & AC = 0xFFFF$\And$A = A \\\hline
           07D &    2080 & AND 0x080    & AC = A$\And$D \\\hline
           07E &    E081 & ST 0x081     & Сохранение промеж. рез.:\\
               &         &              & R = A$\And$D \\\hline
           07F &    0100 & HLT          & Отключение ТГ,\\
               &         &              & переход в пульт. реж. \\\hline
           080 &    0100 & --           & Промеж. данные:\\
               &         &              & D = C -- B \\\hline
           081 &    6076 & --           & Результат:\\
               &         &              & R = A$\And$D = A$\And$(C -- B)\\\hline
           082 &    2080 & --           & Данные (C) \\\hline
    \end{tabular}
    }
    \caption{Описание данной программы}
    \label{tab:program}
\end{table}


\Subsection{Расположение в памяти ЭВМ программы, исходных данных и результатов}

Программа в памяти ЭВМ распологается в ячейках памяти с адресами 0x077 -- 0x07F. Исходные данные находятся по адресам 0x075, 0x076, 0x082. Результат программы находится по адресу 0x081. 

\Subsection{Адреса первой и последней выполняемой команд программы}
Адрес первой выполняемой команды -- 0x077. Адрес последней выполняемой команды (HLT) -- 0x7F. 

\Subsection{Область представления исходных данных и результата}
A, R -- набор из 16 однобитовых значений. B, С -- знаковые 16-ти разрядные числа.  Результат (A + C) -- набор из 16 однобитовых значений.

\Subsection{Область допустимых значений исходных данных и результата}
% \begin{equation}
% \text{Установим свойство: }
% \lt[N\leqslant \text{\textasciicircum}X \leqslant M\rt] \LRarr 
% \lt[\text{\textasciicircum}M\leqslant X \leqslant \text{\textasciicircum}N\rt]\label{eq:not}
% \end{equation}


$$
A\And(C-B)\Rarr
\begin{array}{cc}
 & \boxed{A_i\in \lbrace 1, 0 \rbrace, \text{ где } 0\leqslant i \leqslant 15};\\[2mm]
& \boxed{X=(C-B),\; X_i\in \lbrace 1, 0 \rbrace, \text{ где } 0\leqslant i \leqslant 15};
\end{array}\xRightarrow{\text{Сл. } 1^\circ, 2^\circ, 3^\circ}
$$\\[2mm]


% $\lt[-2^{15}\leqslant C-B\leqslant2^{15}-1\rt] 
% \LRarr
% \lt[-2^{15}\leqslant C + (\text{\textasciicircum} B + 1)\leqslant2^{15}-1 \rt] \xRightarrow{\text{Сл. } 1^\circ, 2^\circ, 3^\circ}$\\[2mm]


\textbf{$1^\circ.$} (Смотри рисунок \ref{fig:way1})

$$
\begin{cases}
-2^{14}\leqslant C\leqslant2^{14}-1\\
-2^{14}\leqslant -B\leqslant2^{14}-1\\
\end{cases} 
\xLeftrightarrow{} 
\begin{cases}
-2^{14}\leqslant C\leqslant2^{14}-1\\
-2^{14}\leqslant B\leqslant2^{14}\\
\end{cases} 
$$\\[5mm]

\textbf{$2^\circ.$} (Смотри рисунок \ref{fig:way2})

$$
\begin{cases}
-2^{15}\leqslant C\leqslant 0\\
0\leqslant -B\leqslant2^{15}-1\\
\end{cases} 
\xLeftrightarrow{} 
\begin{cases}
-2^{15}\leqslant C\leqslant 0\\
-2^{15}+1\leqslant B\leqslant 0\\
\end{cases} 
$$\\[5mm]

\textbf{$3^\circ.$} (Смотри рисунок \ref{fig:way3})

$$
\begin{cases}
0 \leqslant C\leqslant 2^{15}-1\\
-2^{15}\leqslant -B\leqslant 0\\
\end{cases} 
\xLeftrightarrow{} 
\begin{cases}
0 \leqslant C\leqslant 2^{15}-1\\
0 \leqslant B\leqslant 2^{15}\\
\end{cases} 
$$\\[5mm]

Таким образом итоговое ОДЗ:\\

$$
\begin{cases}
A_i\in \lbrace 1, 0 \rbrace, \text{ где } 0\leqslant i \leqslant 15\\
\lt[
\begin{array}{l}
\begin{cases}
-2^{14}\leqslant C\leqslant2^{14}-1\\
-2^{14}\leqslant B\leqslant2^{14}\\
\end{cases} \\[1mm]
\begin{cases}
-2^{15}\leqslant C\leqslant 0\\
-2^{15}+1\leqslant B\leqslant 0\\
\end{cases}\\[1mm]
\begin{cases}
0 \leqslant C\leqslant 2^{15}-1\\
0 \leqslant B\leqslant 2^{15}\\
\end{cases} 
\end{array}
\rt.
\end{cases}
$$

\incfig{lab-2-fopa-ODZ-1}{Случай 1.}{way1}{0.9\textwidth}
\incfig{lab-2-fopa-ODZ-2}{Случай 2.}{way2}{0.9\textwidth}
\incfig{lab-2-fopa-ODZ-3}{Случай 3.}{way3}{0.9\textwidth}

\Section{Таблица трассироваки}
Промежуточные данные после выполнения каждой команды программы представлены в таблице трассировки \ref{tab:trace}.
\begin{table}[htbp]
    \renewcommand{\arraystretch}{1.2} % Установка высоты строки таблицы
    \centering
    \resizebox{\columnwidth}{!}{%
    \begin{tabular}{*{13}{|c}|}
        \hline
        \multicolumn{2}{|c|}{Выполняемая} & \multicolumn{8}{c|}{Содержимое регистров процессора после} & \multicolumn{2}{c|}{Ячейка, содержимое}\\
        \multicolumn{2}{|c|}{команда}     & \multicolumn{8}{c|}{выполнения команды.} & \multicolumn{2}{c|}{которой изменилось после}\\
        \multicolumn{2}{|c|}{}&\multicolumn{8}{c|}{}&\multicolumn{2}{c|}{выполнения команды}\\\hline
        Адрес & Знчение & IP & CR & AR & DR & SP & BR & AC & NZVC & Адрес & Значение\\\hline
        077 & A082 & 077 & 0000 & 000 & 0000 & 000 & 0000  & 0000 & 0100 &   & \\
        077 & A082 & 078 & A082 & 082 & 2080 & 000 & 0077  & 2080 & 0000 &   & \\
        078 & 6076 & 079 & 6076 & 076 & A082 & 000 & 0078  & 7FFE & 0000 &   & \\
        079 & E080 & 07A & E080 & 080 & 7FFE & 000 & 0079  & 7FFE & 0000 & 080 & 7FFE\\
        07A & 0200 & 07B & 0200 & 07A & 0200 & 000 & 007A  & 0000 & 0100 &  &\\
        07B & 0280 & 07C & 0280 & 07B & 0280 & 000 & 007B  & FFFF & 1000 &  &\\
        07C & 2075 & 07D & 2075 & 075 & 2075 & 000 & 007C  & 2075 & 0000 &  &\\
        07D & 2080 & 07E & 2080 & 080 & 7FFE & 000 & 007D  & 2074 & 0000 &  &\\
        07E & E081 & 07F & E081 & 081 & 2074 & 000 & 007E  & 2074 & 0000 & 081 & 2074\\
        07F & 0100 & 080 & 0100 & 07F & 0100 & 000 & 007F  & 2074 & 0000 &  &\\ \hline
    \end{tabular}
    }
    \caption{Таблица трассировки.}
    \label{tab:trace}
\end{table}



\Section{Вариант программы с меньшим числом команд}
Описание программы с меньшим числом команд представлено в таблице \ref{tab:optimized}.
Смысловую часть программы можно изобразить формулой:\\ $R=A\And(C-B)$. Программа считает разность переменных $C$ и $B$, затем логически умножает полученное значение на $A$ и записывает полученное значение в первую свободную ячейку памяти (сразу после команд программы).

\begin{table}[htbp]
    \renewcommand{\arraystretch}{1.2} % Установка высоты строки таблицы
    \centering
    \begin{tabular}{|l|l|l|l|}
            \hline
           Адрес, (0x)   & Данные/      & Мнемоника & Результат\\
                         & /Команда, (0x)&           &          \\\hline
           000 &    2075 & --           & Данные (A)\\\hline
           001 &    A082 & --           & Данные (B)\\\hline
           002 &    2080 & --           & Данные (C)\\\hline
           003 &    A002 & LD 0x002     & AC = C \\\hline
           004 &    6001 & SUB 0x001    & AC = AC - B \\\hline
           005 &    2000 & AND 0x000    & AC = A$\And$AC = A$\And$(C -- B)\\\hline
           006 &    E008 & ST 0x008     & Cохранение рез.: \\
               &         &              & 0x008: A$\And$(C -- B)\\\hline
           007 &    0100 & HLT          & Отключение ТГ,\\
               &         &              & переход в пульт. реж. \\\hline
    \end{tabular}
    \caption{Описание сокращенной программы}
    \label{tab:optimized}
\end{table}





\Section{Вывод}
В ходе выполнения лабораторной работы было изучено устройство БЭВМ, основные её команды. На  эмуляторе БЭВМ была запущена простейшая программа и произведён разбор состояний БЭВМ при  пошаговом выполнении программы.\\
\newpage
%<<<<<<<<<<<<<<<<<<<<<< КОД РАБОТЫ <<<<<<<<<<<<<<<<<<<<<<<<


%>>>>>>>>>>>>>>>> СПИСОК ЛИТЕРАТУРЫ >>>>>>>>>>>>>>>>>>>>>>>
%\include{biblist}  % Для соответсвия гост, придется доработать. Нужен файл .bib
%<<<<<<<<<<<<<<<<<<<< СПИСОК ЛИТЕРАТУРЫ <<<<<<<<<<<<<<<<<<<


\end{document}
%<<<<<<<<<<<<<<<< ,,,,,,,,,,,,,,,,,,,,,,, <<<<<<<<<<<<<<<<<
%<<<<<<<<<<<<<<<<<<< СОДЕРЖИМОЕ ОТЧЕТА <<<<<<<<<<<<<<<<<<<<
