\begin{tabular}{|l|l|l|l|}

\hline
\textbf{Адрес} & \textbf{\begin{tabular}[c]{@{}l@{}}Данные/\\ /Команда\end{tabular}} & \textbf{Мнемоника}                                                             & \textbf{Описание [, метка]}                                                                                                       \\ \hline
\hex{569}      & \hex{057e}                                                          & -                                                                              & Адрес начала массива \ttt{Y0}, \ttt{X1:}                                                                                          \\ \hline
\hex{56a}      & \hex{a000}                                                          & -                                                                              & \begin{tabular}[c]{@{}l@{}}Данные затираются значением \\ \ttt{X1=Y0}, по ним происходит \\ итерирование, \ttt{X2:}\end{tabular}  \\ \hline
\hex{56b}      & \hex{e000}                                                          & -                                                                              & \begin{tabular}[c]{@{}l@{}}Данные затираются значением \\ \hex{04}, количество элементов\\ массива, \ttt{X3:}\end{tabular}        \\ \hline
\hex{56c}      & \hex{0200}                                                          & -                                                                              & \begin{tabular}[c]{@{}l@{}}Данные обнуляются, счетчик\\ чисел кратных 4, \ttt{X4:}\end{tabular}                                   \\ \hline
\hex{56d}      & \hex{0200}                                                          & \ttt{cla}                                                                      & Очистка аккумулятора                                                                                                              \\ \hline
\hex{56e}      & \hex{eefd}                                                          & \ttt{st X4}                                                                    & Запись \ttt{AC} в \ttt{X4}                                                                                                        \\ \hline
\hex{56f}      & \hex{af04}                                                          & \ttt{ld \#\hex{04}}                                                            & Чтение в \ttt{AC} числа \hex{04}                                                                                                  \\ \hline
\hex{570}      & \hex{eefa}                                                          & \ttt{st X3}                                                                    & Запись \ttt{AC} в \ttt{X3}                                                                                                        \\ \hline
\hex{571}      & \hex{aef7}                                                          & \ttt{ld X1}                                                                    & Чтение в \ttt{AC} значения \ttt{X1}                                                                                               \\ \hline
\hex{572}      & \hex{eef7}                                                          & \ttt{st X2}                                                                    & Запись \ttt{AC} в \ttt{X2}                                                                                                        \\ \hline
\hex{573}      & \hex{aaf6}                                                          & \ttt{ld (X2)+}                                                                 & \begin{tabular}[c]{@{}l@{}}Автоинкрементное чтение в \\ \ttt{AC} из \ttt{X2}, и начало\\ цикла, \ttt{\_loop:}\end{tabular}         \\ \hline
\hex{574}      & \hex{0480}                                                          & \ttt{ror}                                                                      & \begin{tabular}[c]{@{}l@{}}Цикличиский сдвиг вправо, \\ \ttt{C} $=AC\ \mathrm{mod}\ 2$\end{tabular}                                 \\ \hline
\hex{575}      & \hex{f405}                                                          & \begin{tabular}[c]{@{}l@{}}\ttt{bhis \_endl}\\ (\ttt{blo \_endl})\end{tabular}  & \begin{tabular}[c]{@{}l@{}}Переход к концу цикла, если \\ \ttt{AC=Yi} не делится на 2\\ (\ttt{C=1})\end{tabular}                  \\ \hline
\hex{576}      & \hex{0480}                                                          & \ttt{ror}                                                                      & \begin{tabular}[c]{@{}l@{}}Цикличиский сдвиг вправо, \\ \ttt{C} $=AC\ \mathrm{mod}\ 2$\end{tabular}                                 \\ \hline
\hex{577}      & \hex{f403}                                                          & \begin{tabular}[c]{@{}l@{}}\ttt{bhis \_endl}\\ (\ttt{blo \_endl})\end{tabular} & \begin{tabular}[c]{@{}l@{}}Переход к концу цикла, если \\ \ttt{AC=Yi} не делится на 4 \\ (\ttt{C=1})\end{tabular}                 \\ \hline
\hex{578}      & \hex{0400}                                                          & \ttt{rol}                                                                      & Цикличиский сдвиг влево                                                                                                           \\ \hline
\hex{579}      & \hex{0400}                                                          & \ttt{rol}                                                                      & Цикличиский сдвиг влево                                                                                                           \\ \hline
\hex{57a}      & \hex{6af1}                                                          & \ttt{sub (X4)+}                                                                & \begin{tabular}[c]{@{}l@{}}Вычитание из \ttt{AC} значения \\ по адресу в \ttt{X4}, сдвиг указателя\\ \ttt{X4} вперед\end{tabular} \\ \hline
\hex{57b}      & \hex{856b}                                                          & \ttt{loop \$X3}                                                                & \begin{tabular}[c]{@{}l@{}}Уменьшение значения в \ttt{X3} до\\  нуля в цикле, конец цикла, \ttt{\_endl:}\end{tabular}             \\ \hline
\hex{57c}      & \hex{cef6}                                                          & \ttt{jump \_loop}                                                              & Переход к началу цикла \ttt{\_loop}                                                                                               \\ \hline
\hex{57d}      & \hex{0100}                                                          & \ttt{hlt}                                                                      & Останов                                                                                                                           \\ \hline
\hex{57e}      & \hex{0100}                                                          & -                                                                              & Первый элемент массива, \ttt{Y0:}                                                                                                 \\ \hline
\hex{57f}      & \hex{0683}                                                          & -                                                                              & Второй элемент массива                                                                                                            \\ \hline
\hex{580}      & \hex{0c00}                                                          & -                                                                              & Третий элемент массива                                                                                                            \\ \hline
\hex{581}      & \hex{f200}                                                          & -                                                                              & Четвертый элемент массива                                                                                                         \\ \hline
\end{tabular}
