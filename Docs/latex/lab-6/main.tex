%%%%%%%%%%%%%%%%%%%%%%%%%%%%%%%%% LAB-5 %%%%%%%%%%%%%%%%%%%%%%%%%%%%%%%%%%
%>>>>>>>>>>>>>>>>>>>>>>>>>> ПЕРЕМЕННЫЕ >>>>>>>>>>>>>>>>>>>>>>>>>>>>>>>>>>>
%>>>>> Информация о кафедре
%\newcommand{\year}{2021 г.}  % Год устанавливается автоматически
\newcommand{\city}{Санкт-Петербург}  %  Футер, нижний колонтитул на титульном листе
\newcommand{\university}{Национальный исследовательский университет ИТМО}  % первая строка
\newcommand{\department}{Факультет программной инженерии и компьютерной техники}  % Вторая строка
\newcommand{\major}{Направление системного и прикладного программного обеспечения}  % Треьтя строка
%<<<<< Информация о кафедре

%>>>>> Назание работы
\newcommand{\reporttype}{ОТЧЕТ ПО ЛАБОРАТОРНОЙ РАБОТЕ} % тип работы, (главный заголовок титульного листа)
\newcommand{\lab}{Лабораторная работа}          % вид работы
\newcommand{\labnumber}{№ 5}                    % порядковый номер работы
\newcommand{\subject}{Основы профессиональной деятельности}         % учебный предмет
\newcommand{\labtheme}{Исследование работы БЭВМ: асинхронный ввод/вывод с помощью прерываний}            % Тема лабораторной работы
\newcommand{\variant}{№ 1007}                % номер варианта работы

\newcommand{\student}{Тюрин Иван Николаевич}    % определение ФИО студента
\newcommand{\studygroup}{P3110}                 % определение учебной группы 
\newcommand{\teacher}{Клименков С. В.,\\[1mm]     % ФИО лектора
                        Ларочкин Г. И.}          % ФИО практика
%<<<<<<<<<<<<<<<<<<<<<<<<<< ПЕРЕМЕННЫЕ <<<<<<<<<<<<<<<<<<<<<<<<<<<<<<<<<<<


%>>>>>>>>>>>>>>>>>>>>>> ПРЕАМБУЛА >>>>>>>>>>>>>>>>>>>>>>>>>
\include{preamble}
%<<<<<<<<<<<<<<<<<<<<<< ПРЕАМБУЛА <<<<<<<<<<<<<<<<<<<<<<<<<



%%%%%%%%%%%%%%%%%%% СОДЕРЖИМОЕ ОТЧЕТА %%%%%%%%%%%%%%%%%%%%%
%>>>>>>>>>>>>>>> ''''''''''''''''''''''' >>>>>>>>>>>>>>>>>>
\begin{document}


%>>>>>>>>>>>>>>>> ОПРЕДЕЛЕНИЕ НАЗВАНИЙ >>>>>>>>>>>>>>>>>>>>
% Переоформление некоторых стандартных названий
%\renewcommand{\chaptername}{Лабораторная работа}
\renewcommand{\chaptername}{\lab\ \labnumber} % переименование глав
\def\contentsname{Содержание} % переименование оглавления
%<<<<<<<<<<<<<<<< ОПРЕДЕЛЕНИЕ НАЗВАНИЙ <<<<<<<<<<<<<<<<<<<<


%>>>>>>>>>>>>>>>>> ТИТУЛЬНАЯ СТРАНИЦА >>>>>>>>>>>>>>>>>>>>>
\include{titlepage}
%<<<<<<<<<<<<<<<<< ТИТУЛЬНАЯ СТРАНИЦА <<<<<<<<<<<<<<<<<<<<<


%>>>>>>>>>>>>>>>>>>>>> СОДЕРЖАНИЕ >>>>>>>>>>>>>>>>>>>>>>>>>
% Содержание
\tableofcontents
%<<<<<<<<<<<<<<<<<<<<< СОДЕРЖАНИЕ <<<<<<<<<<<<<<<<<<<<<<<<<


%%%%%%%%%%%%%%%%%%%%%%% КОД РАБОТЫ %%%%%%%%%%%%%%%%%%%%%%%%
%>>>>>>>>>>>>>>>>>>>'''''''''''''''''>>>>>>>>>>>>>>>>>>>>>
\newpage
\Chapter{\lab\ \labnumber}{\labtheme}{}

\Section{Задание варианта \variant}
\begin{center}
, , ,
\end{center}
\noindent





\textit{По выданному преподавателем варианту разработать и исследовать работу комплекса программ обмена данными в режиме прерывания программы. Основная программа должна изменять содержимое заданной ячейки памяти (Х), которое должно быть представлено как знаковое число. Область допустимых значений изменения Х должна быть ограничена заданной функцией F(X) и конструктивными особенностями регистра данных ВУ (8-ми битное знаковое представление). Программа обработки прерывания должна выводить на ВУ модифицированное значение Х в соответствии с вариантом задания, а также игнорировать все необрабатываемые прерывания.}
\begin{enumerate}
    \item Основная программа должна уменьшать на 2 содержимое $X$ (ячейки памяти с адресом $015_{16}$) в цикле.
    \item Обработчик прерывания должен по нажатию кнопки готовности. \textit{ВУ-3} осуществлять вывод результата вычисления функции $F(X)=3X+1$ на данное ВУ, a по нажатию кнопки готовности \textit{ВУ-2} прибавить утроенное содержимое РД данного ВУ к $X$, результат записать в $X$.
    \item Если $Х$ оказывается вне ОДЗ при выполнении любой операции по его изменению, то необходимо в $Х$ записать максимальное по ОДЗ число.
\end{enumerate}
\begin{center}
    ' ' '
\end{center}

\newpage
\Section{Описание программы}

\Subsection{Назначение программы}

Описание программы представлено в таблице \ref{tab:program-desc}.
Программа осуществляет асинхронный ввод с \textit{ВУ-2} и вывод на \textit{ВУ-3} при помощи прерываний. \textit{ВУ-2} производит прерывание при выставлении готовности кнопкой, после чего над переменной в программе производятся определенные операции. \textit{ВУ-3} производит прерывание при выставлении готовности кнопкой, после чего над переменной производятся операции и результат выводится на табло этого устройства.

% \Table{program-desc}{Описание работы подпрограмм}{
    %\begin{longtable}{!t}
    \begin{longtable}{|l|l|l|l|}
\hline
\textbf{Адрес} & \textbf{Метка} & \textbf{Мнемоника} &\textbf{Описание} 
\\ \hline
\hex{000}	&		&	org		0x000	&		\\	
\hex{001}	&	v0:	&	word		\$default, 0x180	&	Инициализация вектора 	\\	
\hex{002}	&	v1:	&	word		\$default, 0x180	&	прерываний	\\	
\hex{003}	&	v2:	&	word		\$int2, 0x180	&		\\	
\hex{004}	&	v3:	&	word		\$int3, 0x180	&		\\	
\hex{005}	&	v4:	&	word		\$default, 0x180	&		\\	
\hex{006}	&	v5:	&	word		\$default, 0x180	&		\\	
\hex{007}	&	v6:	&	word		\$default, 0x180	&		\\	
\hex{008}	&	v7:	&	word		\$default, 0x180	&		\\	\hline
			&	&	og		0x015	&	Параметры программы	\\	
\hex{015}	&	x:	&	word		?	&	Переменная (X)	\\	
\hex{016}	&	uplim:	&	word		0x002a	&	Верхняя граница ОДЗ	\\	
\hex{017}	&	lowlim:	&	word		0xffd5	&	Нижняя граница ОДЗ	\\	
\hex{018}	&	default:	&	iret			&	Стандарт. обработка прер.	\\	\hline
\hex{019}	&	int2:	&	nop			&	Обработчик прерывания ВУ-2, 	\\	
\hex{01A}	&		&	cla			&	точка для остановы	\\	
\hex{01B}	&		&	in		0x4	&	Считывние из ВУ-2	\\	
\hex{01C}	&		&	sxtb			&	Прибавление утроенного 	\\	
\hex{01D}	&		&	push			&	значания  РД ВУ-2	\\	
\hex{01E}	&		&	asl			&		\\	
\hex{01F}	&		&	add    		\&0	&		\\	
\hex{020}	&		&	add		x	&		\\	
\hex{021}	&		&	call		\_check	&		\\	
\hex{022}	&		&	st		x	&		\\	
\hex{023}	&		&	pop			&		\\	
\hex{024}	&		&	ld		x	&		\\	
\hex{025}	&		&	nop			&	Точка для остановы	\\	
\hex{026}	&		&	iret			&		\\	\hline
\hex{027}	&	int3:	&	nop			&	Обработчик прерывания ВУ-3, 	\\	
\hex{028}	&		&	ld		x	&	точка для остановы	\\	
\hex{029}	&		&	asl			&	Арифметическая операция 3x+1	\\	
\hex{02A}	&		&	add		x	&		\\	
\hex{02B}	&		&	inc			&		\\	
\hex{02C}	&		&	out		0x6	&	Вывод на ВУ-3	\\	
\hex{02D}	&		&	nop			&	Точка для остановы	\\	
\hex{02E}	&		&	iret			&		\\	\hline
\hex{02F}	&	\_check:	&	cmp uplim			&	Подпрограмма для проверки	\\	
\hex{030}	&		&	bpl \_setmax			&	 выхода из ОДЗ	\\	
\hex{031}	&		&	cmp lowlim			&		\\	
\hex{032}	&		&	bmi \_setmax			&		\\	
\hex{033}	&	\_retcheck:	&	ret			&		\\	
\hex{034}	&	\_setmax:	&	ld uplim			&		\\	
\hex{035}	&		&	jump \_retcheck			&		\\	\hline
\hex{036}	&	START:	&	di			&	Начало программы, 	\\	
\hex{037}	&		&	cla			&	инициализация векторов 	\\	
\hex{038}	&		&	out		0x1	&	прерывания	\\	
\hex{039}	&		&	out		0x3	&		\\	
\hex{03A}	&		&	out		0xb	&		\\	
\hex{03B}	&		&	out		0xf	&		\\	
\hex{03C}	&		&	out		0x13	&		\\	
\hex{03D}	&		&	out		0x17	&		\\	
\hex{03E}	&		&	out		0x1b	&		\\	
\hex{03F}	&		&	out		0x1f	&		\\	
\hex{040}	&		&	ld		\#0xa	&		\\	
\hex{041}	&		&	out		0x5	&		\\	
\hex{042}	&		&	ld		\#0xb	&		\\	
\hex{043}	&		&	out		0x7	&		\\	
\hex{044}	&		&	ei			&		\\	\hline
\hex{045}	&	main:	&	di			&	Главный цикл	\\	
\hex{046}	&		&	ld		x	&	Уменьшение переменной на 2	\\	
\hex{047}	&		&	sub		\#2	&		\\	
\hex{048}	&		&	call		\_check	&		\\	
\hex{049}	&		&	st		x	&		\\	
\hex{04A}	&		&	ei			&		\\	
\hex{04B}	&		&	nop			&		\\	
\hex{04С}	&		&	jump		main	&		\\	
\hline
\caption{Описание работы программ}
\label{tab:program-desc}
\end{longtable} %%%%%%%%%%% ТАБЛИЦА %%%%%%%%%%
    %\end{longtable}
% }

\Subsection{Область допустимых значений}
Пользователю доступны для ввода любые однобайтовые знаковые числа, то есть из диапозона $[-128;127]$. ОДЗ для переменной $X$ находится из соотношения:
$$-128\leq 3x+1 \leq 127 \Rarr $$
$$\hex{ffd5}=-43 \leq x \leq 42 = \hex{002a}$$

\Section{Методика проверки}
\textbf{Главный цикл}
\begin{enumerate}
    \item В переменную x записать \hex{ffd8}.
\item Поставить точку остановы в цикле main на адрес \hex{04b}.
\item Загрузить программный компекс в память БЭВМ.
\item Запустить выполнение программы в автоматическом режиме с адреса \hex{036} до остановки.
\item Записать значение аккумулятора, оно должно быть на 2 меньше, чем число которое мы записали в переменную.
\item Запустить выполнение программы с адреса предыдущей остановки до слудующей остановки.
\item Записать значение аккумулятора и сверить его с предпологаемым: оно должно быть равно верхней границе ОДЗ \hex{002a} (т.к. на новом витке цикла переменная вышла из ОДЗ).
\end{enumerate}

\textbf{Обработчик прерывания ВУ-2}
\begin{enumerate}
\item Убрать прежние и поставить точки остановы в подпрограмме \verb|int2| на адресах \hex{019} и \hex{025}.
\item Загрузить программный компекс в память БЭВМ.
\item Запустить выполнение программы в автоматическом режиме с адреса \hex{036}.
\item Установить значение и выставить режим готовности на ВУ-2.
\item Выполнять программу до точки остановы.
\item Записать значение аккумулятора.
\item Выполнить программу до следующей точки остановы.
\item Записать значение аккумулятора.
\item Рассчитать значение переменной по формуле $x+3y$, где $y$ - число введенное на ВУ-2, если вычисленное значение не находится в ОДЗ, то должно быть равно \hex{02a}.
\item Сравнить полученные значения аккумулятора: второе должно удовлетворять рассчетному.
\end{enumerate}

\textbf{Обработчик прерывания ВУ-3}
\begin{enumerate}
\item Убрать прежние и поставить точки остановы в подпрограмме int3 на адресах \hex{027} и \hex{02d}.
\item Загрузить программный комплекс в память БЭВМ.
\item Запустить выполнение программы в автоматическом режиме с адреса \hex{036}.
\item Выставить режим готовности на ВУ-3.
\item Выполнять программу до точки остановы.
\item Записать значение аккумулятора.
\item Выполнить программу до следующей точки остановы.
\item Записать значение аккумулятора.
\item Рассчитать значение переменной по формуле $3x+1$, если вычисленное значение не находится в ОДЗ, то должно быть равно \hex{02a}.
\item Сравнить полученные значения аккумулятора: второе должно удовлетворять рассчетному.
\end{enumerate}


\Section{Программа}
Была составлена программа на языке ассемблера. Она представлена в листингaх \ref{lst:script-p1} и \ref{lst:script-p2}.

\refstepcounter{lstlisting}
\begin{figure}[h] %- \usepackage {float} %[h]
    \begin{center}
        \lstinputlisting[]{listings/script-1part.asm}
    \end{center}
    \captionof{lstlisting}{Первая чать кода программы на языке ассемблена БЭВМ: параметры программы, переменные, инициализация векторов прерывания, обработчики прерывания}
    \label{lst:script-p1}
\end{figure}

\refstepcounter{lstlisting}
\begin{figure}[h] %- \usepackage {float} %[h]
    \begin{center}
        \lstinputlisting[]{listings/script-2part.asm}
    \end{center}
    \captionof{lstlisting}{Вторая часть кода программы на языке ассемблена БЭВМ: подпрограмма проверки выхода результата вычисленяи из ОДЗ и главный цикл}
    \label{lst:script-p2}
\end{figure}


\Section{Вывод}
Нацчился работать с прерываниями в БЭВМ, писать код на языке ассемблера БЭВМ для их обработки. Научился делать одновременно несколько важных лаб за ночь и терпению в очереди за день.

\newpage
%<<<<<<<<<<<<<<<<<<<<<< КОД РАБОТЫ <<<<<<<<<<<<<<<<<<<<<<<<


%>>>>>>>>>>>>>>>> СПИСОК ЛИТЕРАТУРЫ >>>>>>>>>>>>>>>>>>>>>>>
%\include{biblist}  % Для соответсвия гост, придется доработать. Нужен файл .bib
%<<<<<<<<<<<<<<<<<<<< СПИСОК ЛИТЕРАТУРЫ <<<<<<<<<<<<<<<<<<<


\end{document}
%<<<<<<<<<<<<<<<< ,,,,,,,,,,,,,,,,,,,,,,, <<<<<<<<<<<<<<<<<
%<<<<<<<<<<<<<<<<<<< СОДЕРЖИМОЕ ОТЧЕТА <<<<<<<<<<<<<<<<<<<<
