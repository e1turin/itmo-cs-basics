%%%%%%%%%%%%%%%%%%%%%%%%%%%%%%%%% LAB-5 %%%%%%%%%%%%%%%%%%%%%%%%%%%%%%%%%%
%>>>>>>>>>>>>>>>>>>>>>>>>>> ПЕРЕМЕННЫЕ >>>>>>>>>>>>>>>>>>>>>>>>>>>>>>>>>>>
%>>>>> Информация о кафедре
%\newcommand{\year}{2021 г.}  % Год устанавливается автоматически
\newcommand{\city}{Санкт-Петербург}  %  Футер, нижний колонтитул на титульном листе
\newcommand{\university}{Национальный исследовательский университет ИТМО}  % первая строка
\newcommand{\department}{Факультет программной инженерии и компьютерной техники}  % Вторая строка
\newcommand{\major}{Направление системного и прикладного программного обеспечения}  % Треьтя строка
%<<<<< Информация о кафедре

%>>>>> Назание работы
\newcommand{\reporttype}{ОТЧЕТ ПО ЛАБОРАТОРНОЙ РАБОТЕ} % тип работы, (главный заголовок титульного листа)
\newcommand{\lab}{Лабораторная работа}          % вид работы
\newcommand{\labnumber}{№ 5}                    % порядковый номер работы
\newcommand{\subject}{Основы профессиональной деятельности}         % учебный предмет
\newcommand{\labtheme}{Исследование работы БЭВМ: асинхронный ввод/вывод с помощью прерываний}            % Тема лабораторной работы
\newcommand{\variant}{№ 1007}                % номер варианта работы

\newcommand{\student}{Тюрин Иван Николаевич}    % определение ФИО студента
\newcommand{\studygroup}{P3110}                 % определение учебной группы 
\newcommand{\teacher}{Клименков С. В.,\\[1mm]     % ФИО лектора
                        Ларочкин Г. И.}          % ФИО практика
%<<<<<<<<<<<<<<<<<<<<<<<<<< ПЕРЕМЕННЫЕ <<<<<<<<<<<<<<<<<<<<<<<<<<<<<<<<<<<


%>>>>>>>>>>>>>>>>>>>>>> ПРЕАМБУЛА >>>>>>>>>>>>>>>>>>>>>>>>>

%>>>>>>>>>>>>>>>>>> ПРЕАМБУЛА >>>>>>>>>>>>>>>>>>>>
\documentclass[14pt,final,oneside]{extreport}% класс документа, характеристики
%>>>>> Разметка документа
\usepackage[a4paper, mag=1000, left=3cm, right=1.5cm, top=1cm, bottom=2cm, headsep=0.7cm, footskip=1cm]{geometry} % По ГОСТу: left>=3cm, right=1cm, top=2cm, bottom=2cm,
\linespread{1} % межстройчный интервал по ГОСТу := 1.5
%<<<<< Разметка документа

%>>>>> babel c языковым пакетом НЕ должны быть первым импортируемым пакетом
\usepackage[utf8]{inputenc}
\usepackage[T1,T2A]{fontenc}
\usepackage[russian]{babel}
%<<<<<

%\usepackage{cmap} %поиск в pdf

%>>>...>> прочие полезные пакеты
\usepackage{amsmath,amsthm,amssymb}
\usepackage{mathtext}
\usepackage{indentfirst}
\usepackage{graphicx}
\usepackage{float}
\graphicspath{{/home/ivan/itmo/informatics/latex}}
\DeclareGraphicsExtensions{.pdf,.png,.jpg}
%\usepackage{bookmark}

\usepackage[dvipsnames]{xcolor}
\usepackage{hyperref}  % Использование ссылок
\hypersetup{%  % Настройка разметки ссылок
    colorlinks=true,
    linkcolor=blue,
    filecolor=magenta,      
    urlcolor=magenta,
    %pdftitle={Overleaf Example},
    %pdfpagemode=FullScreen,
}
\usepackage{longtable}
\usepackage{diagbox}
\usepackage[letterspace=150]{microtype} % Спэйсинг (межбуквенный интервал для саголовка) \lsstyle

%>>> верстка в 2 колонки
\usepackage{multicol} % многоколоночная верстка
\setlength{\columnsep}{.15\textwidth} % определение ширины разделителя между колонками

%> кастомный разделитель колонок
\usepackage{tikz} % пакет для векторной графики, чтобы рисовать красивый разделитель колонок
\usetikzlibrary{arrows.meta,decorations.pathmorphing,backgrounds,positioning,fit,petri}
\usepackage{multicolrule} % Для кастомизации разделителя колонок
\SetMCRule{                     % кастомизация разделителя колонок multicolrule
    width=2pt,
    custom-line={               % Tikz код для кастомизации линии разделителя
        \draw [                 % Рисовать
            decorate,           % декорированную (требуются спец настройки пакетов tikz (см. импорт выше)
            decoration={        % вид декорирования
                snake, % Тип - змейка (волнистая)
                amplitude=.5mm, % ширина волн
                pre length=0mm, % участок прямой линии от начала
                %segment length=0mm, % учасок волнистой линии
                post length=0mm % участок прямой линии от конца
            },
            line width=1pt,
            step=10pt
        ] 
        (TOP) to (BOT); % сверху и до низа колонки
    }, 
    extend-top=-5pt, % Вылезти за верхнюю границу колонки 
    extend-bot=-7pt % Вылезти за нижнюю границу колонки  
}
%< кастомный разделитель колонок
%<<< верстка в 2 колонки

%>>>>> Использование листингов
\usepackage{listings} 
\usepackage{caption}
\DeclareCaptionFont{white}{\color{white}} 
\DeclareCaptionFormat{listing}{\colorbox{gray}{\parbox{\textwidth}{#1#2#3}}}

\captionsetup[lstlisting]{format=listing,labelfont=white,textfont=white} % Настройка вида описаний
\lstset{  % Настройки вида листинга
inputencoding=utf8, extendedchars=\true, keepspaces = true, % поддержка кириллицы и пробелов в комментариях
language={},            % выбор языка для подсветки (здесь это Pascal)
basicstyle=\small\sffamily, % размер и начертание шрифта для подсветки кода
numbers=left,               % где поставить нумерацию строк (слева\справа)
numberstyle=\tiny,          % размер шрифта для номеров строк
stepnumber=1,               % размер шага между двумя номерами строк
numbersep=5pt,              % как далеко отстоят номера строк от подсвечиваемого кода
backgroundcolor=\color{white}, % цвет фона подсветки - используем \usepackage{color}
showspaces=false,           % показывать или нет пробелы специальными отступами
showstringspaces=false,     % показывать илигнет пробелы в строках
showtabs=false,             % показывать или нет табуляцию в строках
frame=single,               % рисовать рамку вокруг кода
tabsize=2,                  % размер табуляции по умолчанию равен 2 пробелам
captionpos=t,               % позиция заголовка вверху [t] или внизу [b] 
breaklines=true,            % автоматически переносить строки (да\нет)
breakatwhitespace=false,    % переносить строки только если есть пробел
escapeinside={\%*}{*)}      % если нужно добавить комментарии в коде
}
%<<<<< Использование листингов
\usepackage{csvsimple} %импорт содержимого таблицы из csv


\sloppy % Решение проблем с переносами (с. 119 книга Львовского)
\emergencystretch=25pt


%>>>>>>>>>>>>>>>> ДОПОЛНИТЕЛЬНЫЕ КОМАНДЫ {Для соответствия ГОСТ} >>>>>>>>>>>>>>
%>>>>>> математические функции для удобства
\newcommand{\eps}{\varepsilon}
\newcommand{\limit}{\displaystyle\lim}
\newcommand{\oo}{\infty}
\newcommand{\Dx}{\Delta x}
\newcommand{\cd}{\cdot}
% \newcommand{\note}[2]{\overbrace{#1}^{#2}} % скобка сверху для комментария

%>>>>> Скобки
\newcommand{\lt}{\left}
\newcommand{\rt}{\right}
%<<<<< Скобки

%>>>>> Дроби
\newcommand{\cf}{\cfrac}
\newcommand{\fr}{\frac}
%<<<<< Дроби


%>>>>> Стрелки
\newcommand{\Rarr}{\Rightarrow}% ⇒ следствие
\newcommand{\LRarr}{\Leftrightarrow}% равносильно
\newcommand{\rarr}{\xrightarrow{}}% → стрелка вправо
\newcommand{\nwarr}{\nwarrow}% ↖ север-запад стрелка
\newcommand{\nearr}{\nearrow}% ↗ север-восток стрелка
\newcommand{\swarr}{\swarrow}% ↙ юг-запад стрелка
\newcommand{\searr}{\searrow}% ↘ юг-восток стрелка

\newcommand{\raises}{\nwarrow}% возрастает
\newcommand{\falls}{\swarrow}% убывает
\newcommand{\increases}{\nwarrow}% возрастает
\newcommand{\decreases}{\swarrow}% убывает

%{{{
\makeatletter
\newcommand{\xLeftrightarrow}[2][]{\ext@arrow 0359\Leftrightarrowfill@{#1}{#2}}% следствие с надписью
\makeatother
%}}}

%{{{
\makeatletter
\newcommand{\xRightarrow}[2][]{\ext@arrow 0359\Rightarrowfill@{#1}{#2}}% равносильность с надписью
\makeatother
%}}}
%<<<<< Стрелки
%>>>>> Стиль текста
\newcommand{\hex}[1]{\texttt{0{\footnotesize{x}}#1}}
\newcommand{\ttt}[1]{\texttt{#1}}
%<<<<< Стиль текста

%<<<<<< математические функции для удобства

\newcommand\Chapter[3]{%
    % Принимает 3 аргумента - название главы и дополнительный заголовок и ширина загловка (можно ничего)
    \refstepcounter{chapter}%
    \chapter*{%
        %\hfill % заполнение отступом пространства до заголовка
        \begin{minipage}{#3\textwidth} % Можно изменить ширину министраницы (заголовка)
            \flushleft % Выранивание заголовка по левому краю параграфа (заголовка)
            %\flushright % Выранивание заголовка по правому краю параграфа (заголовка)
            \begin{huge}%
                % Отключена нумерация глав в тексте:
                %:=% \textbf{\chaptername\ \arabic{chapter}\\}
                \textbf{#1}% Первый заголовок
            \end{huge}%
            \\% Перенос сторки
            \begin{Huge}
                #2% Второй заголовок
            \end{Huge}
        \end{minipage}
    }%
    % Отключена нумерация для chapter в toc (table of contents), т.е. Оглавлении (Содержании):
    %:=% \addcontentsline{toc}{chapter}{\arabic{chapter}. #1}
    % Представление главы в содержании:
    \addcontentsline{toc}{chapter}{#1. #2}%
}

\newcommand\Section[1]{
    % Принимает 1 аргумент - название секции
    \refstepcounter{section}
    \section*{%
        \raggedright
        % Отключена дополнительная нумерация chapter в section в тексте документа:
        %:=% \arabic{chapter}.\arabic{section}. #1}
        % Отключена любая нумарация section в тексте документа: (убрать \arabic{section}, оставить название секции)
        \arabic{section}. #1
    }
    
    % Отключена дополнительная нумерация chapter в section в toc (table of contents) Оглавлении (Содержании):
    %:=% \addcontentsline{toc}{section}{\arabic{chapter}.\arabic{section}. #1}
    \addcontentsline{toc}{section}{\arabic{section}. #1} 
}


\newcommand\Subsection[1]{
    % Принимает 1 аргумент - название подсекции
    \refstepcounter{subsection}
    \subsection*{%
        \raggedright%
        % Отключена дополнительная нумерация chapter в section в тексте документа (можно добавить отступ с помощью \hspace*{12pt}):
        %:=% \arabic{chapter}.\arabic{section}.\arabic{subsection}. #1}
        \arabic{section}. \arabic{subsection}. #1
    }
    % Отключена дополнительная нумерация chapter в section в Оглавлении (Содержании):
    %\addcontentsline{toc}{subsection}{\arabic{chapter}.\arabic{section}.\arabic{subsection}. #1}
    \addcontentsline{toc}{subsection}{\arabic{subsection}. #1}
}


\newcommand\Figure[4]{
    % Принимает 4 аргумента - название файла изображения, ее размер в тексте, описание, лэйбл (псевдоним в формате "fig:name") 
    %
    \refstepcounter{figure}
    \begin{figure}[H] %- \usepackage {float} %[h]
        \begin{center}
            \includegraphics[width=#2]{#1}
        \end{center}
        \caption{#3}
        \label{fig:#4}
    \end{figure}
}


\newcommand\Table[3]{
    % Принимает 3 аргумента --- лэйбл name(#1) (псевдоним в формате "tab:name"), ее описание(#2), содержание таблицы(#3) 
    %
    \refstepcounter{table}
    \renewcommand{\arraystretch}{1} % Установка высоты строки таблицы по умолчанию (=1), увеличенное на 0.2 пункта
    % \refstepcounter{table}% увеличение счетчика таблиц
    \begin{table}[htpb]% "right here"=H, "top", "new page", "bottom"
        \label{tab:#1}% лэйбл таблицы, для ссылок
        \resizebox{\columnwidth}{!}{% сжимает очень широкие таблицы, чтобы вместить на страницу
             #3% Содержание таблицы
        }
        % 
        \caption{#2}% Описание стандартными средствами для используемого окружения (table)
        % \captionof{table}{#2}% Описание стандартными средствами
        % \captionof*{figure}{\flushleft \textsc\textbf{Рис. 1.}}% Описание стандартными средствами, как рисунка
        %
        %%> кастомное описание
        % \begin{flushleft}% Кастомное описание
        %     % \textsf{%
        %         \textbf{%
        %             \\[2mm]
        %             #2% Описание к картинке
        %         }%
        %         % \\[8mm]% Отступ
        %     % }%
        % \end{flushleft}
        %%< кастомное описание
    \end{table}
    \renewcommand{\arraystretch}{1} % возврат установка высоты строки таблицы по умолчанию на 1
}


\newcommand\CustomFigure[4]{ % multicols не умеют в table и figure, поэтому приходится извращаться % вставка таблицы с меткой рисунка
    % Принимает 4 аргумента - название файла изображения, ее размер в тексте, описание, лэйбл (псевдоним в формате "fig:name") 
    %
    \refstepcounter{figure}
    \begin{figure}[ht]% "here", "top"
        \begin{center}
            \includegraphics[width=#2]{#1}
        \end{center}
        %
        %\caption{#3}
        \captionof{figure}{#3}% описание стандартными средствами
        % \begin{center}
        \begin{flushleft} % Кастомное описание
            \textbf{%
                #3% Текст описания
            }
        \end{flushleft}
        % \end{center}
        %
        \label{fig:#4}% Лэйбл, для ссылок
    \end{figure}
}


\newcommand\CustomTableFigure[3]{ % multicols не умеют в table и figure, поэтому приходится извращаться % вставка таблицы с меткой рисунка
    %
    % Принимает 3 аргумента --- лэйбл name(#1) (псевдоним в формате "tab:name"), ее описание(#2), содержание таблицы(#3) 
    %
    \begin{center}
        \refstepcounter{figure}
        \label{tab:#1}% лэйбл таблицы, для ссылок
        \resizebox{\columnwidth}{!}{% сжимает очень широкие таблицы, чтобы вместить на страницу
            #3% Содержание таблицы
        }
        % 
        \captionof{figure}{#2}% Описание стандартными средствами
        % \captionof*{figure}{\flushleft \textsc\textbf{Рис. 1.}}% Описание стандартными средствами
        %
        \begin{flushleft}% Кастомное описание
            % \textsf{%
                \textbf{%
                    \\[2mm]
                    #2% Описание к картинке
                }%
                % \\[8mm]% Отступ
            % }%
        \end{flushleft}
    \end{center}
}


\newcommand{\InkscapeFigure}[4]{% Вставки иллюстраций из Inkscape (pdf+latex)
    %
    % Принимает 4 параметра: #1 название файла, #2 описание, #3 лейбл #4 размер
    %
    % \begin{minipage}{#4}
        \begin{figure}[htbp]
            \centering
            \def\svgwidth{#4}
            \import{./figures/}{#1.pdf_tex}
            \caption{#2}
            \label{fig:#3}
        \end{figure}
    % \end{minipage}
}


\newcommand\Equation[3]{% Кастомное оформление выражений
    %
    % Принимает 3 аргумента --- лэйбл name (#1) (псевдоним в формате "tab:name"), его описание(#2), содержание выражения (#3) 
    %
    \textbf{#2}% описание
    \begin{equation}
        #3% содержимое выражений
        \label{eq:#1}% лэйбл
    \end{equation}
}

%<<<<<<<<<<<<<<<<<<<<<<<<<<<< ДОПОЛНИТЕЛЬНЫЕ КОМАНДЫ <<<<<<<<<<<<<<<<<<<<<<<<<<

%<<<<<<<<<<<<<<<<<<<<<< ПРЕАМБУЛА <<<<<<<<<<<<<<<<<<<<<<<<<



%%%%%%%%%%%%%%%%%%% СОДЕРЖИМОЕ ОТЧЕТА %%%%%%%%%%%%%%%%%%%%%
%>>>>>>>>>>>>>>> ''''''''''''''''''''''' >>>>>>>>>>>>>>>>>>
\begin{document}


%>>>>>>>>>>>>>>>> ОПРЕДЕЛЕНИЕ НАЗВАНИЙ >>>>>>>>>>>>>>>>>>>>
% Переоформление некоторых стандартных названий
%\renewcommand{\chaptername}{Лабораторная работа}
\renewcommand{\chaptername}{\lab\ \labnumber} % переименование глав
\def\contentsname{Содержание} % переименование оглавления
%<<<<<<<<<<<<<<<< ОПРЕДЕЛЕНИЕ НАЗВАНИЙ <<<<<<<<<<<<<<<<<<<<


%>>>>>>>>>>>>>>>>> ТИТУЛЬНАЯ СТРАНИЦА >>>>>>>>>>>>>>>>>>>>>
%>>>>>>>>>>>>>>>>>>> ТИТУЛЬНЫЙ ЛИСТ >>>>>>>>>>>>>>>>>>>>>>>
\begin{titlepage}

    % Название университета
    \begin{center}
    \textsc{%
        \university\\[5mm]
        \department\\[2mm]
        \major\\
    }

    \vfill
    \vfill
    % Название работы
    \textbf{\reporttype\ \labnumber\\[3mm]
    курса <<\subject>> \\[6mm]
    по теме: <<\labtheme>>\\[3mm]
    Вариант \variant\\[20mm]
    }
    \end{center}


\hfill
% Информация об авторе работы и проверяющем
\begin{minipage}{.5\textwidth}
    \begin{flushright}
        
            
        Выполнил студент:\\[2mm] 
        \student\\[2mm]
        группа: \studygroup\\[5mm]

        Преподаватель:\\[2mm] 
        \teacher

    \end{flushright}
\end{minipage}

\vfill

    % Нижний колонтитул первой страницы
    \begin{center}
        \city, \the\year\,г.
    \end{center}

\end{titlepage}
%<<<<<<<<<<<<<<<<<<< ТИТУЛЬНЫЙ ЛИСТ <<<<<<<<<<<<<<<<<<<<<<<


%<<<<<<<<<<<<<<<<< ТИТУЛЬНАЯ СТРАНИЦА <<<<<<<<<<<<<<<<<<<<<


%>>>>>>>>>>>>>>>>>>>>> СОДЕРЖАНИЕ >>>>>>>>>>>>>>>>>>>>>>>>>
% Содержание
\tableofcontents
%<<<<<<<<<<<<<<<<<<<<< СОДЕРЖАНИЕ <<<<<<<<<<<<<<<<<<<<<<<<<


%%%%%%%%%%%%%%%%%%%%%%% КОД РАБОТЫ %%%%%%%%%%%%%%%%%%%%%%%%
%>>>>>>>>>>>>>>>>>>>'''''''''''''''''>>>>>>>>>>>>>>>>>>>>>
\newpage
\Chapter{\lab\ \labnumber}{\labtheme}{}

\Section{Задание варианта \variant}
\begin{center}
, , ,
\end{center}
\noindent





\textit{По выданному преподавателем варианту разработать и исследовать работу комплекса программ обмена данными в режиме прерывания программы. Основная программа должна изменять содержимое заданной ячейки памяти (Х), которое должно быть представлено как знаковое число. Область допустимых значений изменения Х должна быть ограничена заданной функцией F(X) и конструктивными особенностями регистра данных ВУ (8-ми битное знаковое представление). Программа обработки прерывания должна выводить на ВУ модифицированное значение Х в соответствии с вариантом задания, а также игнорировать все необрабатываемые прерывания.}
\begin{enumerate}
    \item Основная программа должна уменьшать на 2 содержимое $X$ (ячейки памяти с адресом $015_{16}$) в цикле.
    \item Обработчик прерывания должен по нажатию кнопки готовности. \textit{ВУ-3} осуществлять вывод результата вычисления функции $F(X)=3X+1$ на данное ВУ, a по нажатию кнопки готовности \textit{ВУ-2} прибавить утроенное содержимое РД данного ВУ к $X$, результат записать в $X$.
    \item Если $Х$ оказывается вне ОДЗ при выполнении любой операции по его изменению, то необходимо в $Х$ записать максимальное по ОДЗ число.
\end{enumerate}
\begin{center}
    ' ' '
\end{center}

\newpage
\Section{Описание программы}

\Subsection{Назначение программы}

Описание программы представлено в таблице \ref{tab:program-desc}.
Программа осуществляет асинхронный ввод с \textit{ВУ-2} и вывод на \textit{ВУ-3} при помощи прерываний. \textit{ВУ-2} производит прерывание при выставлении готовности кнопкой, после чего над переменной в программе производятся определенные операции. \textit{ВУ-3} производит прерывание при выставлении готовности кнопкой, после чего над переменной производятся операции и результат выводится на табло этого устройства.

% \Table{program-desc}{Описание работы подпрограмм}{
    %\begin{longtable}{!t}
    \begin{tabular}{|l|l|l|l|}
\hline
\textbf{Адрес} & \textbf{Метка} & \textbf{Мнемоника} & \textbf{Описание} \\ \hline
\hex{5d2	}	&	\ttt{	endchar:	}	&	\ttt{	word 0x0a	}	&	Код стоп символа	\\	
\hex{5d3	}	&	\ttt{	mask:	}	&	\ttt{	word 0x00ff	}	&	Маска для отделения младшего байта слова	\\	
\hex{5d4	}	&	\ttt{	str:	}	&	\ttt{	word 0x562	}	&	Адрес начала строки	\\	
\hex{5d5	}	&	\ttt{	str.len:	}	&	\ttt{	word 0x0	}	&	Длина строки	\\	
\hex{5d6	}	&	\ttt{	str.iter:	}	&	\ttt{	word 0x0	}	&	Переменная итерирования по строке	\\	
\hex{5d7	}	&	\ttt{	char:	}	&	\ttt{	word 0x0	}	&	Переменная для кода полученного символа	\\	
\hex{5d8	}	&	\ttt{	START:	}	&	\ttt{	ld  \$str 	}	&	Начало программы	\\	
\hex{5d9	}	&	\ttt{		}	&	\ttt{	st \$str.iter	}	&	Итератор по строке устанавливается в начало	\\	\hline
\hex{5da	}	&	\ttt{	\_readstr:  	}	&	\ttt{	cla 	}	&	Цикл ввода строки	\\	
\hex{5db	}	&	\ttt{		}	&	\ttt{	call    \_readchar 	}	&	Вызов подпрограммы считывания символа	\\	
\hex{5dc	}	&	\ttt{		}	&	\ttt{	st \$char	}	&		\\	
\hex{5dd	}	&	\ttt{		}	&	\ttt{	ld \$str.len	}	&	Увеличение длины строки на 1	\\	
\hex{5de	}	&	\ttt{		}	&	\ttt{	inc 	}	&		\\	
\hex{5df	}	&	\ttt{		}	&	\ttt{	st \$str.len	}	&		\\	
\hex{5e0	}	&	\ttt{		}	&	\ttt{	ror 	}	&		\\	
\hex{5e1	}	&	\ttt{		}	&	\ttt{	bcc \_oddpos      	}	&	Обработка позиции символа в троке	\\	
\hex{5e2	}	&	\ttt{		}	&	\ttt{	ld \$char	}	&		\\	
\hex{5e3	}	&	\ttt{		}	&	\ttt{	st (str.iter)	}	&		\\	
\hex{5e4	}	&	\ttt{		}	&	\ttt{	jump \_is\_end	}	&		\\	
\hex{5e5	}	&	\ttt{	\_oddpos:	}	&	\ttt{	ld \$char	}	&	Изменение положения кода символа в слове для четных позиций	\\	
\hex{5e6	}	&	\ttt{		}	&	\ttt{	swab 	}	&		\\	
\hex{5e7	}	&	\ttt{		}	&	\ttt{	add (str.iter)	}	&		\\	
\hex{5e8	}	&	\ttt{		}	&	\ttt{	st (str.iter)+	}	&	Сдвиг итератора на позицию следующей пары символов	\\	
\hex{5e9	}	&	\ttt{	\_is\_end:             	}	&	\ttt{	ld \$char	}	&	Проверка на стоп-символ	\\	
\hex{5ea	}	&	\ttt{		}	&	\ttt{	cmp endchar	}	&		\\	
\hex{5eb	}	&	\ttt{		}	&	\ttt{	bne \_readstr	}	&		\\	
\hex{5ec	}	&	\ttt{		}	&	\ttt{	ld \$str	}	&	Итератор устанавливается на начало строки	\\	
\hex{5ed	}	&	\ttt{		}	&	\ttt{	st \$str.iter  	}	&		\\	\hline
\hex{5ee	}	&	\ttt{	\_printstr:	}	&	\ttt{	ld (str.iter)     	}	&	Цикл вывода строки	\\	
\hex{5ef	}	&	\ttt{		}	&	\ttt{	call \_writechar	}	&	Вызов подпрограммы для вывода первого символа в слове	\\	
\hex{5f0	}	&	\ttt{		}	&	\ttt{	ld      (str.iter) 	}	&		\\	
\hex{5f1	}	&	\ttt{		}	&	\ttt{	and     mask 	}	&		\\	
\hex{5f2	}	&	\ttt{		}	&	\ttt{	cmp     endchar 	}	&	Проверка на стоп-символ	\\	
\hex{5f3	}	&	\ttt{		}	&	\ttt{	beq     \_\_stop 	}	&		\\	
\hex{5f4	}	&	\ttt{		}	&	\ttt{	ld      (str.iter)  	}	&		\\	
\hex{5f5	}	&	\ttt{		}	&	\ttt{	swab 	}	&		\\	
\hex{5f6	}	&	\ttt{		}	&	\ttt{	call \_writechar	}	&	Вызов подпрограммы для вывода второго символа в слове	\\	
\hex{5f7	}	&	\ttt{		}	&	\ttt{	ld      (str.iter)+ 	}	&		\\	
\hex{5f8	}	&	\ttt{		}	&	\ttt{	swab 	}	&		\\	
\hex{5f9	}	&	\ttt{		}	&	\ttt{	and     mask 	}	&		\\	
\hex{5fa	}	&	\ttt{		}	&	\ttt{	cmp     endchar 	}	&		\\	
\hex{5fb	}	&	\ttt{		}	&	\ttt{	bne     \_printstr 	}	&		\\	\hline
\hex{5fc	}	&	\ttt{	\_\_stop:	}	&	\ttt{	hlt 	}	&	Останов	\\	\hline
\hex{5fd	}	&	\ttt{	\_readchar:            	}	&	\ttt{	in      7 	}	&	Подпрограмма для ввода символа	\\	
\hex{5fe	}	&	\ttt{		}	&	\ttt{	and     \#0x40 	}	&	Цикл ожидания ввода	\\	
\hex{5ff	}	&	\ttt{		}	&	\ttt{	beq     \_readchar 	}	&		\\	
\hex{600	}	&	\ttt{		}	&	\ttt{	in      6 	}	&	Ввод кода символа	\\	
\hex{601	}	&	\ttt{		}	&	\ttt{	and     mask 	}	&		\\	
\hex{602	}	&	\ttt{		}	&	\ttt{	ret 	}	&		\\	\hline
\hex{603	}	&	\ttt{	\_writechar:         	}	&	\ttt{	out     0xc 	}	&	Подпрограмма для вывода символа	\\	
\hex{604	}	&	\ttt{	\_waitwrote:	}	&	\ttt{	in 0xe	}	&	Ожидание окончания вывода	\\	
\hex{605	}	&	\ttt{		}	&	\ttt{	ror 	}	&		\\	
\hex{606	}	&	\ttt{		}	&	\ttt{	bcs \_waitwrote	}	&		\\	
\hex{607	}	&	\ttt{		}	&	\ttt{	ret 	}	&		\\	\hline	
\end{tabular} %%%%%%%%%%% ТАБЛИЦА %%%%%%%%%%
    %\end{longtable}
% }

\Subsection{Область допустимых значений}
Пользователю доступны для ввода любые однобайтовые знаковые числа, то есть из диапозона $[-128;127]$. ОДЗ для переменной $X$ находится из соотношения:
$$-128\leq 3x+1 \leq 127 \Rarr $$
$$\hex{ffd5}=-43 \leq x \leq 42 = \hex{002a}$$

\Section{Методика проверки}
\textbf{Главный цикл}
\begin{enumerate}
    \item В переменную x записать \hex{ffd8}.
\item Поставить точку остановы в цикле main на адрес \hex{04b}.
\item Загрузить программный компекс в память БЭВМ.
\item Запустить выполнение программы в автоматическом режиме с адреса \hex{036} до остановки.
\item Записать значение аккумулятора, оно должно быть на 2 меньше, чем число которое мы записали в переменную.
\item Запустить выполнение программы с адреса предыдущей остановки до слудующей остановки.
\item Записать значение аккумулятора и сверить его с предпологаемым: оно должно быть равно верхней границе ОДЗ \hex{002a} (т.к. на новом витке цикла переменная вышла из ОДЗ).
\end{enumerate}

\textbf{Обработчик прерывания ВУ-2}
\begin{enumerate}
\item Убрать прежние и поставить точки остановы в подпрограмме \verb|int2| на адресах \hex{019} и \hex{025}.
\item Загрузить программный компекс в память БЭВМ.
\item Запустить выполнение программы в автоматическом режиме с адреса \hex{036}.
\item Установить значение и выставить режим готовности на ВУ-2.
\item Выполнять программу до точки остановы.
\item Записать значение аккумулятора.
\item Выполнить программу до следующей точки остановы.
\item Записать значение аккумулятора.
\item Рассчитать значение переменной по формуле $x+3y$, где $y$ - число введенное на ВУ-2, если вычисленное значение не находится в ОДЗ, то должно быть равно \hex{02a}.
\item Сравнить полученные значения аккумулятора: второе должно удовлетворять рассчетному.
\end{enumerate}

\textbf{Обработчик прерывания ВУ-3}
\begin{enumerate}
\item Убрать прежние и поставить точки остановы в подпрограмме int3 на адресах \hex{027} и \hex{02d}.
\item Загрузить программный комплекс в память БЭВМ.
\item Запустить выполнение программы в автоматическом режиме с адреса \hex{036}.
\item Выставить режим готовности на ВУ-3.
\item Выполнять программу до точки остановы.
\item Записать значение аккумулятора.
\item Выполнить программу до следующей точки остановы.
\item Записать значение аккумулятора.
\item Рассчитать значение переменной по формуле $3x+1$, если вычисленное значение не находится в ОДЗ, то должно быть равно \hex{02a}.
\item Сравнить полученные значения аккумулятора: второе должно удовлетворять рассчетному.
\end{enumerate}


\Section{Программа}
Была составлена программа на языке ассемблера. Она представлена в листингaх \ref{lst:script-p1} и \ref{lst:script-p2}.

\refstepcounter{lstlisting}
\begin{figure}[h] %- \usepackage {float} %[h]
    \begin{center}
        \lstinputlisting[]{listings/script-1part.asm}
    \end{center}
    \captionof{lstlisting}{Первая чать кода программы на языке ассемблена БЭВМ: параметры программы, переменные, инициализация векторов прерывания, обработчики прерывания}
    \label{lst:script-p1}
\end{figure}

\refstepcounter{lstlisting}
\begin{figure}[h] %- \usepackage {float} %[h]
    \begin{center}
        \lstinputlisting[]{listings/script-2part.asm}
    \end{center}
    \captionof{lstlisting}{Вторая часть кода программы на языке ассемблена БЭВМ: подпрограмма проверки выхода результата вычисленяи из ОДЗ и главный цикл}
    \label{lst:script-p2}
\end{figure}


\Section{Вывод}
Нацчился работать с прерываниями в БЭВМ, писать код на языке ассемблера БЭВМ для их обработки. Научился делать одновременно несколько важных лаб за ночь и терпению в очереди за день.

\newpage
%<<<<<<<<<<<<<<<<<<<<<< КОД РАБОТЫ <<<<<<<<<<<<<<<<<<<<<<<<


%>>>>>>>>>>>>>>>> СПИСОК ЛИТЕРАТУРЫ >>>>>>>>>>>>>>>>>>>>>>>
%
\bibliographystyle{plain}

\begin{thebibliography}{3}
    \addcontentsline{toc}{chapter}{Список лиетратуры}

    \bibitem{gutgut:1}
    Код Хэмминга. Пример работы алгоритма. URL: \url{https://habr.com/ru/post/140611/};

    \bibitem{gutgut:2}
    Избыточное кодирование, код Хэмминга. URL: \url{https://neerc.ifmo.ru/wiki/index.php?title=%D0%98%D0%B7%D0%B1%D1%8B%D1%82%D0%BE%D1%87%D0%BD%D0%BE%D0%B5_%D0%BA%D0%BE%D0%B4%D0%B8%D1%80%D0%BE%D0%B2%D0%B0%D0%BD%D0%B8%D0%B5,_%D0%BA%D0%BE%D0%B4_%D0%A5%D1%8D%D0%BC%D0%BC%D0%B8%D0%BD%D0%B3%D0%B0}.

\end{thebibliography}  % Для соответсвия гост, придется доработать. Нужен файл .bib
%<<<<<<<<<<<<<<<<<<<< СПИСОК ЛИТЕРАТУРЫ <<<<<<<<<<<<<<<<<<<


\end{document}
%<<<<<<<<<<<<<<<< ,,,,,,,,,,,,,,,,,,,,,,, <<<<<<<<<<<<<<<<<
%<<<<<<<<<<<<<<<<<<< СОДЕРЖИМОЕ ОТЧЕТА <<<<<<<<<<<<<<<<<<<<
