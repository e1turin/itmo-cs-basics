%%%%%%%%%%%%%%%%%%%%%%%%%%%%%%%%% LAB-5 %%%%%%%%%%%%%%%%%%%%%%%%%%%%%%%%%%
%>>>>>>>>>>>>>>>>>>>>>>>>>> ПЕРЕМЕННЫЕ >>>>>>>>>>>>>>>>>>>>>>>>>>>>>>>>>>>
%>>>>> Информация о кафедре
%\newcommand{\year}{2021 г.}  % Год устанавливается автоматически
\newcommand{\city}{Санкт-Петербург}  %  Футер, нижний колонтитул на титульном листе
\newcommand{\university}{Национальный исследовательский университет ИТМО}  % первая строка
\newcommand{\department}{Факультет программной инженерии и компьютерной техники}  % Вторая строка
\newcommand{\major}{Направление системного и прикладного программного обеспечения}  % Треьтя строка
%<<<<< Информация о кафедре

%>>>>> Назание работы
\newcommand{\reporttype}{ОТЧЕТ ПО ЛАБОРАТОРНОЙ РАБОТЕ} % тип работы, (главный заголовок титульного листа)
\newcommand{\lab}{Лабораторная работа}          % вид работы
\newcommand{\labnumber}{№ 7}                    % порядковый номер работы
\newcommand{\subject}{Основы профессиональной деятельности}         % учебный предмет
\newcommand{\labtheme}{Исследование работы БЭВМ: микрокоманды и синтез инструкций}            % Тема лабораторной работы
\newcommand{\variant}{№ 1010}                % номер варианта работы

\newcommand{\student}{Тюрин Иван Николаевич}    % определение ФИО студента
\newcommand{\studygroup}{P3110}                 % определение учебной группы 
\newcommand{\teacher}{Клименков С. В.,\\[1mm]     % ФИО лектора
                        Ларочкин Г. И.}          % ФИО практика
%<<<<<<<<<<<<<<<<<<<<<<<<<< ПЕРЕМЕННЫЕ <<<<<<<<<<<<<<<<<<<<<<<<<<<<<<<<<<<


%>>>>>>>>>>>>>>>>>>>>>> ПРЕАМБУЛА >>>>>>>>>>>>>>>>>>>>>>>>>
\include{preamble}
%<<<<<<<<<<<<<<<<<<<<<< ПРЕАМБУЛА <<<<<<<<<<<<<<<<<<<<<<<<<



%%%%%%%%%%%%%%%%%%% СОДЕРЖИМОЕ ОТЧЕТА %%%%%%%%%%%%%%%%%%%%%
%>>>>>>>>>>>>>>> ''''''''''''''''''''''' >>>>>>>>>>>>>>>>>>
\begin{document}


%>>>>>>>>>>>>>>>> ОПРЕДЕЛЕНИЕ НАЗВАНИЙ >>>>>>>>>>>>>>>>>>>>
% Переоформление некоторых стандартных названий
%\renewcommand{\chaptername}{Лабораторная работа}
\renewcommand{\chaptername}{\lab\ \labnumber} % переименование глав
\def\contentsname{Содержание} % переименование оглавления
%<<<<<<<<<<<<<<<< ОПРЕДЕЛЕНИЕ НАЗВАНИЙ <<<<<<<<<<<<<<<<<<<<


%>>>>>>>>>>>>>>>>> ТИТУЛЬНАЯ СТРАНИЦА >>>>>>>>>>>>>>>>>>>>>
\include{titlepage}
%<<<<<<<<<<<<<<<<< ТИТУЛЬНАЯ СТРАНИЦА <<<<<<<<<<<<<<<<<<<<<


%>>>>>>>>>>>>>>>>>>>>> СОДЕРЖАНИЕ >>>>>>>>>>>>>>>>>>>>>>>>>
% Содержание
\tableofcontents
%<<<<<<<<<<<<<<<<<<<<< СОДЕРЖАНИЕ <<<<<<<<<<<<<<<<<<<<<<<<<


%%%%%%%%%%%%%%%%%%%%%%% КОД РАБОТЫ %%%%%%%%%%%%%%%%%%%%%%%%
%>>>>>>>>>>>>>>>>>>>'''''''''''''''''>>>>>>>>>>>>>>>>>>>>>
\newpage
\Chapter{\lab\ \labnumber}{\labtheme}{}

\Section{Задание варианта \variant}
\begin{center}
, , ,
\end{center}
\noindent

\textit{Синтезировать цикл исполнения для выданных преподавателем команд. Разработать тестовые программы, которые проверяют каждую из синтезированных команд. Загрузить в микропрограммную память БЭВМ циклы исполнения синтезированных команд, загрузить в основную память БЭВМ тестовые программы. Проверить и отладить разработанные тестовые программы и микропрограммы.}
\begin{enumerate}
    \item \verb|MSUB M| - вычитание аккумулятора из М с записью результата в ячейку памяти с установкой N/Z/V/C
    \item Код операции - 9...
    \item Тестовая программа должна начинаться с адреса \hex{0481}
\end{enumerate}
\begin{center}
    ' ' '
\end{center}

\newpage

\Section{Описание синтезированной микропрограммы}
В результате выполнения работы была синетзированна следующая микропрограмма, описание которой представлено в таблице \ref{tab:program-desc}.

Запись микропрограммы в код БЭВМ осуществляется следующей командой:\\
\verb|e0 ma mw 0001e09611 mw 8055101040|
\begin{longtable}{|l|l|l|l|}
\hline
\textbf{Адрес} & \textbf{Метка} & \textbf{Мнемоника} &\textbf{Описание} 
\\ \hline
\hex{000}	&		&	org		0x000	&		\\	
\hex{001}	&	v0:	&	word		\$default, 0x180	&	Инициализация вектора 	\\	
\hex{002}	&	v1:	&	word		\$default, 0x180	&	прерываний	\\	
\hex{003}	&	v2:	&	word		\$int2, 0x180	&		\\	
\hex{004}	&	v3:	&	word		\$int3, 0x180	&		\\	
\hex{005}	&	v4:	&	word		\$default, 0x180	&		\\	
\hex{006}	&	v5:	&	word		\$default, 0x180	&		\\	
\hex{007}	&	v6:	&	word		\$default, 0x180	&		\\	
\hex{008}	&	v7:	&	word		\$default, 0x180	&		\\	\hline
			&	&	og		0x015	&	Параметры программы	\\	
\hex{015}	&	x:	&	word		?	&	Переменная (X)	\\	
\hex{016}	&	uplim:	&	word		0x002a	&	Верхняя граница ОДЗ	\\	
\hex{017}	&	lowlim:	&	word		0xffd5	&	Нижняя граница ОДЗ	\\	
\hex{018}	&	default:	&	iret			&	Стандарт. обработка прер.	\\	\hline
\hex{019}	&	int2:	&	nop			&	Обработчик прерывания ВУ-2, 	\\	
\hex{01A}	&		&	cla			&	точка для остановы	\\	
\hex{01B}	&		&	in		0x4	&	Считывние из ВУ-2	\\	
\hex{01C}	&		&	sxtb			&	Прибавление утроенного 	\\	
\hex{01D}	&		&	push			&	значания  РД ВУ-2	\\	
\hex{01E}	&		&	asl			&		\\	
\hex{01F}	&		&	add    		\&0	&		\\	
\hex{020}	&		&	add		x	&		\\	
\hex{021}	&		&	call		\_check	&		\\	
\hex{022}	&		&	st		x	&		\\	
\hex{023}	&		&	pop			&		\\	
\hex{024}	&		&	ld		x	&		\\	
\hex{025}	&		&	nop			&	Точка для остановы	\\	
\hex{026}	&		&	iret			&		\\	\hline
\hex{027}	&	int3:	&	nop			&	Обработчик прерывания ВУ-3, 	\\	
\hex{028}	&		&	ld		x	&	точка для остановы	\\	
\hex{029}	&		&	asl			&	Арифметическая операция 3x+1	\\	
\hex{02A}	&		&	add		x	&		\\	
\hex{02B}	&		&	inc			&		\\	
\hex{02C}	&		&	out		0x6	&	Вывод на ВУ-3	\\	
\hex{02D}	&		&	nop			&	Точка для остановы	\\	
\hex{02E}	&		&	iret			&		\\	\hline
\hex{02F}	&	\_check:	&	cmp uplim			&	Подпрограмма для проверки	\\	
\hex{030}	&		&	bpl \_setmax			&	 выхода из ОДЗ	\\	
\hex{031}	&		&	cmp lowlim			&		\\	
\hex{032}	&		&	bmi \_setmax			&		\\	
\hex{033}	&	\_retcheck:	&	ret			&		\\	
\hex{034}	&	\_setmax:	&	ld uplim			&		\\	
\hex{035}	&		&	jump \_retcheck			&		\\	\hline
\hex{036}	&	START:	&	di			&	Начало программы, 	\\	
\hex{037}	&		&	cla			&	инициализация векторов 	\\	
\hex{038}	&		&	out		0x1	&	прерывания	\\	
\hex{039}	&		&	out		0x3	&		\\	
\hex{03A}	&		&	out		0xb	&		\\	
\hex{03B}	&		&	out		0xf	&		\\	
\hex{03C}	&		&	out		0x13	&		\\	
\hex{03D}	&		&	out		0x17	&		\\	
\hex{03E}	&		&	out		0x1b	&		\\	
\hex{03F}	&		&	out		0x1f	&		\\	
\hex{040}	&		&	ld		\#0xa	&		\\	
\hex{041}	&		&	out		0x5	&		\\	
\hex{042}	&		&	ld		\#0xb	&		\\	
\hex{043}	&		&	out		0x7	&		\\	
\hex{044}	&		&	ei			&		\\	\hline
\hex{045}	&	main:	&	di			&	Главный цикл	\\	
\hex{046}	&		&	ld		x	&	Уменьшение переменной на 2	\\	
\hex{047}	&		&	sub		\#2	&		\\	
\hex{048}	&		&	call		\_check	&		\\	
\hex{049}	&		&	st		x	&		\\	
\hex{04A}	&		&	ei			&		\\	
\hex{04B}	&		&	nop			&		\\	
\hex{04С}	&		&	jump		main	&		\\	
\hline
\caption{Описание работы программ}
\label{tab:program-desc}
\end{longtable} %%%%%%%%%%% ТАБЛИЦА %%%%%%%%%%

\Section{Текст тестовых прогамм}
Была составлена программа на языке ассемблера для проверки работоспособности синтезированной микропрограммы. Она представлена в листингaх  \ref{lst:script}. Микропрограмма протестирована для различных типов адресаци, в листинге приведен лишь способ прямой адресации. Со всеми типами адресации синтезированная инструкция ведет себя корректно, за исключением прямой загрузки, при которой имеет место неопределенное поведение.

\refstepcounter{lstlisting}
\begin{figure}[h] %- \usepackage {float} %[h]
    \begin{center}
        \lstinputlisting[]{listings/testscript-template.asm}
    \end{center}
    \captionof{lstlisting}{Код тестирующей программы}
    \label{lst:script}  
\end{figure}



\Section{Методика проверки}
Для проверки корректности работы микропрограммы при помощи тестовой программмы была разработана следующая методика:
\begin{enumerate}
    \item Записать значениям для меток X и Y соответственно значения аккумулятора и значение ячейки M памяти с которой требуется провести опеацию \verb|MSUB M|.
    \item Записать значениям для меток \verb|expect.val|, \verb|expect.N| соответственно рассчитанные значения ячейки M и флага N.
    \item Запустить тестирующую программу в автоматическом режиме до остановки.
    \item Проверить значения переменных \verb|check.val|, \verb|check.N|, \verb|check.res| по адресам с \hex{487} до \hex{489}: они все, а в частности последняя должны быть равны 1. В противном случае микропрограмма работает некорректно.
\end{enumerate}


\Section{Трассировка разработанной микропрограммы}
Трасировка разработанной микропрограммы с начала цикла выборки инструкции до начала следующего, выполняющей вычитание 6 из ячейки с адресом \hex{000}, где хранится число \hex{0010} представлена в таблице 1.3.
\Table{tracetable}{Трассировка микропрограммы c момента начала цикла выборки инструкции до начала следующего}{
    \begin{tabular}{|*{11}{l|}}
\hline
Адр	&	МК	&	IP	&	CR	&	AR	&	DR	&	SP	&	BR	&	AC	&	NZVC	&	СчМК	\\\hline
01	&	00A0009004	&	002	&	0000	&	002	&	0006	&	000	&	0002	&	0006	&	0000	&	02	\\
02	&	0104009420	&	003	&	0000	&	002	&	9000	&	000	&	0002	&	0006	&	0000	&	03	\\
03	&	0002009001	&	003	&	9000	&	002	&	9000	&	000	&	0002	&	0006	&	0000	&	04	\\
04	&	8109804002	&	003	&	9000	&	002	&	9000	&	000	&	0002	&	0006	&	0000	&	09	\\
09	&	800C404002	&	003	&	9000	&	002	&	9000	&	000	&	0002	&	0006	&	0000	&	0C	\\
0C	&	8024084002	&	003	&	9000	&	002	&	9000	&	000	&	0002	&	0006	&	0000	&	24	\\
24	&	8026804002	&	003	&	9000	&	002	&	9000	&	000	&	0002	&	0006	&	0000	&	25	\\
25	&	814A404002	&	003	&	9000	&	002	&	9000	&	000	&	0002	&	0006	&	0000	&	26	\\
26	&	0080009001	&	003	&	9000	&	000	&	9000	&	000	&	0002	&	0006	&	0000	&	27	\\
27	&	0100000000	&	003	&	9000	&	000	&	0010	&	000	&	0002	&	0006	&	0000	&	28	\\
28	&	813C804002	&	003	&	9000	&	000	&	0010	&	000	&	0002	&	0006	&	0000	&	3C	\\
3C	&	8143204002	&	003	&	9000	&	000	&	0010	&	000	&	0002	&	0006	&	0000	&	3D	\\
3D	&	81E0104002	&	003	&	9000	&	000	&	0010	&	000	&	0002	&	0006	&	0000	&	E0	\\
E0	&	0001E09611	&	003	&	9000	&	000	&	000A	&	000	&	0002	&	0006	&	0001	&	E1	\\
E1	&	8055101040	&	003	&	9000	&	000	&	000A	&	000	&	0002	&	0006	&	0001	&	55	\\
55	&	0200000000	&	003	&	9000	&	000	&	000A	&	000	&	0002	&	0006	&	0001	&	56	\\
56	&	80C4101040	&	003	&	9000	&	000	&	000A	&	000	&	0002	&	0006	&	0001	&	C4	\\
C4	&	80DE801040	&	003	&	9000	&	000	&	000A	&	000	&	0002	&	0006	&	0001	&	C5	\\
C5	&	8001401040	&	003	&	9000	&	000	&	000A	&	000	&	0002	&	0006	&	0001	&	01	\\\hline
\end{tabular}
}


\Section{Вывод}
Научился синтезировать собственные команды для БЭВМ при помощи размещения в памяти машины дополнительных микрокоманд. Писать тесты для программ на языке BASM, будет тяжело с ним теперь расстаться: мы стали очень близкими друзьми, знаем друг друга очень хорошо. Теперь умею думать как БЭВМ и знаю ее клевые режимы (\verb|-Dmode=dual, -Dmode=cli|).
\newpage
%<<<<<<<<<<<<<<<<<<<<<< КОД РАБОТЫ <<<<<<<<<<<<<<<<<<<<<<<<


%>>>>>>>>>>>>>>>> СПИСОК ЛИТЕРАТУРЫ >>>>>>>>>>>>>>>>>>>>>>>
%\include{biblist}  % Для соответсвия гост, придется доработать. Нужен файл .bib
%<<<<<<<<<<<<<<<<<<<< СПИСОК ЛИТЕРАТУРЫ <<<<<<<<<<<<<<<<<<<


\end{document}
%<<<<<<<<<<<<<<<< ,,,,,,,,,,,,,,,,,,,,,,, <<<<<<<<<<<<<<<<<
%<<<<<<<<<<<<<<<<<<< СОДЕРЖИМОЕ ОТЧЕТА <<<<<<<<<<<<<<<<<<<<
