%%%%%%%%%%%%%%%%%%%%%%%%%%%%%%%%% LAB-5 %%%%%%%%%%%%%%%%%%%%%%%%%%%%%%%%%%
%>>>>>>>>>>>>>>>>>>>>>>>>>> ПЕРЕМЕННЫЕ >>>>>>>>>>>>>>>>>>>>>>>>>>>>>>>>>>>
%>>>>> Информация о кафедре
%\newcommand{\year}{2021 г.}  % Год устанавливается автоматически
\newcommand{\city}{Санкт-Петербург}  %  Футер, нижний колонтитул на титульном листе
\newcommand{\university}{Национальный исследовательский университет ИТМО}  % первая строка
\newcommand{\department}{Факультет программной инженерии и компьютерной техники}  % Вторая строка
\newcommand{\major}{Направление системного и прикладного программного обеспечения}  % Треьтя строка
%<<<<< Информация о кафедре

%>>>>> Назание работы
\newcommand{\reporttype}{ОТЧЕТ ПО ЛАБОРАТОРНОЙ РАБОТЕ} % тип работы, (главный заголовок титульного листа)
\newcommand{\lab}{Лабораторная работа}          % вид работы
\newcommand{\labnumber}{№ 5}                    % порядковый номер работы
\newcommand{\subject}{Основы профессиональной деятельности}         % учебный предмет
\newcommand{\labtheme}{Исследование работы БЭВМ}            % Тема лабораторной работы
\newcommand{\variant}{№ 1035}                % номер варианта работы

\newcommand{\student}{Тюрин Иван Николаевич}    % определение ФИО студента
\newcommand{\studygroup}{P3110}                 % определение учебной группы 
\newcommand{\teacher}{Клименков С. В.,\\[1mm]     % ФИО лектора
                        Ларочкин Г. И.}          % ФИО практика
%<<<<<<<<<<<<<<<<<<<<<<<<<< ПЕРЕМЕННЫЕ <<<<<<<<<<<<<<<<<<<<<<<<<<<<<<<<<<<


%>>>>>>>>>>>>>>>>>>>>>> ПРЕАМБУЛА >>>>>>>>>>>>>>>>>>>>>>>>>
\include{preamble}
%<<<<<<<<<<<<<<<<<<<<<< ПРЕАМБУЛА <<<<<<<<<<<<<<<<<<<<<<<<<



%%%%%%%%%%%%%%%%%%% СОДЕРЖИМОЕ ОТЧЕТА %%%%%%%%%%%%%%%%%%%%%
%>>>>>>>>>>>>>>> ''''''''''''''''''''''' >>>>>>>>>>>>>>>>>>
\begin{document}


%>>>>>>>>>>>>>>>> ОПРЕДЕЛЕНИЕ НАЗВАНИЙ >>>>>>>>>>>>>>>>>>>>
% Переоформление некоторых стандартных названий
%\renewcommand{\chaptername}{Лабораторная работа}
\renewcommand{\chaptername}{\lab\ \labnumber} % переименование глав
\def\contentsname{Содержание} % переименование оглавления
%<<<<<<<<<<<<<<<< ОПРЕДЕЛЕНИЕ НАЗВАНИЙ <<<<<<<<<<<<<<<<<<<<


%>>>>>>>>>>>>>>>>> ТИТУЛЬНАЯ СТРАНИЦА >>>>>>>>>>>>>>>>>>>>>
\include{titlepage}
%<<<<<<<<<<<<<<<<< ТИТУЛЬНАЯ СТРАНИЦА <<<<<<<<<<<<<<<<<<<<<


%>>>>>>>>>>>>>>>>>>>>> СОДЕРЖАНИЕ >>>>>>>>>>>>>>>>>>>>>>>>>
% Содержание
\tableofcontents
%<<<<<<<<<<<<<<<<<<<<< СОДЕРЖАНИЕ <<<<<<<<<<<<<<<<<<<<<<<<<


%%%%%%%%%%%%%%%%%%%%%%% КОД РАБОТЫ %%%%%%%%%%%%%%%%%%%%%%%%
%>>>>>>>>>>>>>>>>>>>'''''''''''''''''>>>>>>>>>>>>>>>>>>>>>
\newpage
\Chapter{\lab\ \labnumber}{\labtheme}{}

\Section{Задание варианта \variant}
\begin{center}
, , ,
\end{center}
\noindent

\textit{По выданному преподавателем варианту разработать программу асинхронного обмена данными с внешним устройством. При помощи программы осуществить ввод или вывод информации, используя в качестве подтверждения данных сигнал (кнопку) готовности ВУ.}
\begin{itemize}
    \item Программа осуществляет асинхронный ввод данных с \textit{ВУ-3}
    \item Программа начинается с адреса $\mathrm{5D2}_{16}$. Размещаемая строка находится по адресу $562_{16}$.
    \item Строка должна быть представлена в кодировке \textit{КОИ-8}.
    \item Формат представления строки в памяти: \textit{АДР1: СИМВ2 СИМВ1 АДР2: СИМВ4 СИМВ3 ... СТОП\_СИМВ}.
    \item Ввод или вывод строки должен быть завершен по символу c кодом \hex{0A} (NL). Стоп символ является обычным символом строки и подчиняется тем же правилам расположения в памяти что и другие символы строки.
\end{itemize}

\begin{center}
    ' ' '
\end{center}

\newpage
\Section{Описание программы}

\Subsection{Назначение программы}

Описание программы представлено в таблице 1.1. \ref{tab:program-desc}.
Программа осуществляет асинхронный ввод кодов символов (кодировка \textit{KOI8-R})  через \textit{ВУ-3}, и после ввода стоп-символа \hex{0a=NL} прекращает считывание, начинает асинхронный вывод символов соответствующих введенным кодам через \textit{ВУ-5} (текстовый принтер).
\Table{program-desc}{Описание работы подпрограмм}{
    %\begin{longtable}{!t}
    \begin{longtable}{|l|l|l|l|}
\hline
\textbf{Адрес} & \textbf{Метка} & \textbf{Мнемоника} &\textbf{Описание} 
\\ \hline
\hex{000}	&		&	org		0x000	&		\\	
\hex{001}	&	v0:	&	word		\$default, 0x180	&	Инициализация вектора 	\\	
\hex{002}	&	v1:	&	word		\$default, 0x180	&	прерываний	\\	
\hex{003}	&	v2:	&	word		\$int2, 0x180	&		\\	
\hex{004}	&	v3:	&	word		\$int3, 0x180	&		\\	
\hex{005}	&	v4:	&	word		\$default, 0x180	&		\\	
\hex{006}	&	v5:	&	word		\$default, 0x180	&		\\	
\hex{007}	&	v6:	&	word		\$default, 0x180	&		\\	
\hex{008}	&	v7:	&	word		\$default, 0x180	&		\\	\hline
			&	&	og		0x015	&	Параметры программы	\\	
\hex{015}	&	x:	&	word		?	&	Переменная (X)	\\	
\hex{016}	&	uplim:	&	word		0x002a	&	Верхняя граница ОДЗ	\\	
\hex{017}	&	lowlim:	&	word		0xffd5	&	Нижняя граница ОДЗ	\\	
\hex{018}	&	default:	&	iret			&	Стандарт. обработка прер.	\\	\hline
\hex{019}	&	int2:	&	nop			&	Обработчик прерывания ВУ-2, 	\\	
\hex{01A}	&		&	cla			&	точка для остановы	\\	
\hex{01B}	&		&	in		0x4	&	Считывние из ВУ-2	\\	
\hex{01C}	&		&	sxtb			&	Прибавление утроенного 	\\	
\hex{01D}	&		&	push			&	значания  РД ВУ-2	\\	
\hex{01E}	&		&	asl			&		\\	
\hex{01F}	&		&	add    		\&0	&		\\	
\hex{020}	&		&	add		x	&		\\	
\hex{021}	&		&	call		\_check	&		\\	
\hex{022}	&		&	st		x	&		\\	
\hex{023}	&		&	pop			&		\\	
\hex{024}	&		&	ld		x	&		\\	
\hex{025}	&		&	nop			&	Точка для остановы	\\	
\hex{026}	&		&	iret			&		\\	\hline
\hex{027}	&	int3:	&	nop			&	Обработчик прерывания ВУ-3, 	\\	
\hex{028}	&		&	ld		x	&	точка для остановы	\\	
\hex{029}	&		&	asl			&	Арифметическая операция 3x+1	\\	
\hex{02A}	&		&	add		x	&		\\	
\hex{02B}	&		&	inc			&		\\	
\hex{02C}	&		&	out		0x6	&	Вывод на ВУ-3	\\	
\hex{02D}	&		&	nop			&	Точка для остановы	\\	
\hex{02E}	&		&	iret			&		\\	\hline
\hex{02F}	&	\_check:	&	cmp uplim			&	Подпрограмма для проверки	\\	
\hex{030}	&		&	bpl \_setmax			&	 выхода из ОДЗ	\\	
\hex{031}	&		&	cmp lowlim			&		\\	
\hex{032}	&		&	bmi \_setmax			&		\\	
\hex{033}	&	\_retcheck:	&	ret			&		\\	
\hex{034}	&	\_setmax:	&	ld uplim			&		\\	
\hex{035}	&		&	jump \_retcheck			&		\\	\hline
\hex{036}	&	START:	&	di			&	Начало программы, 	\\	
\hex{037}	&		&	cla			&	инициализация векторов 	\\	
\hex{038}	&		&	out		0x1	&	прерывания	\\	
\hex{039}	&		&	out		0x3	&		\\	
\hex{03A}	&		&	out		0xb	&		\\	
\hex{03B}	&		&	out		0xf	&		\\	
\hex{03C}	&		&	out		0x13	&		\\	
\hex{03D}	&		&	out		0x17	&		\\	
\hex{03E}	&		&	out		0x1b	&		\\	
\hex{03F}	&		&	out		0x1f	&		\\	
\hex{040}	&		&	ld		\#0xa	&		\\	
\hex{041}	&		&	out		0x5	&		\\	
\hex{042}	&		&	ld		\#0xb	&		\\	
\hex{043}	&		&	out		0x7	&		\\	
\hex{044}	&		&	ei			&		\\	\hline
\hex{045}	&	main:	&	di			&	Главный цикл	\\	
\hex{046}	&		&	ld		x	&	Уменьшение переменной на 2	\\	
\hex{047}	&		&	sub		\#2	&		\\	
\hex{048}	&		&	call		\_check	&		\\	
\hex{049}	&		&	st		x	&		\\	
\hex{04A}	&		&	ei			&		\\	
\hex{04B}	&		&	nop			&		\\	
\hex{04С}	&		&	jump		main	&		\\	
\hline
\caption{Описание работы программ}
\label{tab:program-desc}
\end{longtable}
    %\end{longtable}
}
\Subsection{Область представления и допустимых значений}
Пользователю доступны для ввода любые однобайтовые беззнаковые числа, то есть из диапозона $[0;255]$, соответствующие символам в кодировке \textit{KOI8-R}, любое из них будет выведено при помощи принтера. Однако после символа с номером $10_{10}$, то есть \hex{0a}, ввод прекращается. Так же при вводе символа с кодом \hex{00} произойдет очистка вывода принтера.

Количество введенных символов не должно превышать $112_{10}=\hex{5D2}-\hex{562}$ (с учетом стоп-символа). Это ограничение вызвано расположением программы и символов строки в памяти указаным в задании.

\Subsection{Трассировка программы}

Трассировка программы для введенных символов \textit{Z, V}, коды которых соответственно \hex{5a} и \hex{56}, представлена в таблице 1.4.\ref{tab:tracetable-in} и 1.6.\ref{tab:tracetable-in}.

Были введены коды символов \textit{Z, V} и код стоп-символа \verb|\n| эти же символы были напечатаны на принтере.

\Table{tracetable-in}{Трассировка программы c с момента старта до момента начала цикла вывода}{
    \begin{tabular}{|*{14}{l|}}
\hline
\multicolumn{2}{|l|}{\begin{tabular}[c]{@{}l@{}}Выполняемая\\команда\end{tabular}} & \multicolumn{9}{l|}{\begin{tabular}[c]{@{}l@{}}Содержимое регистров процессора\\ послевыполнения команды.\end{tabular}} & \multicolumn{2}{l|}{\begin{tabular}[c]{@{}l@{}}Ячейка, содержимое\\которой изменилось\\после выполнения команды\end{tabular}}  \\
\hline
\textbf{Адрес}	& \textbf{Значение}	& \textbf{IP}	& \textbf{CR}	& \textbf{AR}	& \textbf{DR}	&\textbf{SP}	& 	 \textbf{BR}	& \textbf{AC}& \textbf{PS} & \textbf{NZVC}	& \textbf{Адрес}	& \textbf{Значение}\\\hline
5D8 & A5D4 & 5D8 & 0000 & 000 & 0000 & 000 & 0000 & 0000 & 004 & 0100 & &\\
5D8 & A5D4 & 5D9 & A5D4 & 5D4 & 0562 & 000 & 05D8 & 0562 & 000 & 0000 & &\\
5D9 & E5D6 & 5DA & E5D6 & 5D6 & 0562 & 000 & 05D9 & 0562 & 000 & 0000 & 5D6 & 0562\\
5DA & 0200 & 5DB & 0200 & 5DA & 0200 & 000 & 05DA & 0000 & 004 & 0100 & &\\
5DB & DE21 & 5FD & DE21 & 7FF & 05DC & 7FF & 05FD & 0000 & 004 & 0100 & 7FF & 05DC\\
5FD & 1207 & 5FE & 1207 & 5FD & 1207 & 7FF & 05FD & 0040 & 004 & 0100 & &\\
5FE & 2F40 & 5FF & 2F40 & 5FE & 0040 & 7FF & 0040 & 0040 & 000 & 0000 & &\\
5FF & F0FD & 600 & F0FD & 5FF & F0FD & 7FF & 05FF & 0040 & 000 & 0000 & &\\
600 & 1206 & 601 & 1206 & 600 & 1206 & 7FF & 0600 & 005A & 000 & 0000 & &\\
601 & 2ED1 & 602 & 2ED1 & 5D3 & 00FF & 7FF & FFD1 & 005A & 000 & 0000 & &\\
602 & 0A00 & 5DC & 0A00 & 7FF & 05DC & 000 & 0602 & 005A & 000 & 0000 & &\\
5DC & E5D7 & 5DD & E5D7 & 5D7 & 005A & 000 & 05DC & 005A & 000 & 0000 & 5D7 & 005A\\
5DD & A5D5 & 5DE & A5D5 & 5D5 & 0000 & 000 & 05DD & 0000 & 004 & 0100 & &\\
5DE & 0700 & 5DF & 0700 & 5DE & 0700 & 000 & 05DE & 0001 & 000 & 0000 & &\\
5DF & E5D5 & 5E0 & E5D5 & 5D5 & 0001 & 000 & 05DF & 0001 & 000 & 0000 & 5D5 & 0001\\
5E0 & 0480 & 5E1 & 0480 & 5E0 & 0480 & 000 & 05E0 & 0000 & 007 & 0111 & &\\
5E1 & F503 & 5E2 & F503 & 5E1 & F503 & 000 & 05E1 & 0000 & 007 & 0111 & &\\
5E2 & A5D7 & 5E3 & A5D7 & 5D7 & 005A & 000 & 05E2 & 005A & 001 & 0001 & &\\
5E3 & E8F2 & 5E4 & E8F2 & 562 & 005A & 000 & FFF2 & 005A & 001 & 0001 & 562 & 005A\\
5E4 & CE04 & 5E9 & CE04 & 5E4 & 05E9 & 000 & 0004 & 005A & 001 & 0001 & &\\
5E9 & A5D7 & 5EA & A5D7 & 5D7 & 005A & 000 & 05E9 & 005A & 001 & 0001 & &\\
5EA & 7EE7 & 5EB & 7EE7 & 5D2 & 000A & 000 & FFE7 & 005A & 001 & 0001 & &\\
5EB & F1EE & 5DA & F1EE & 5EB & F1EE & 000 & FFEE & 005A & 001 & 0001 & &\\
5DA & 0200 & 5DB & 0200 & 5DA & 0200 & 000 & 05DA & 0000 & 005 & 0101 & &\\
5DB & DE21 & 5FD & DE21 & 7FF & 05DC & 7FF & 05FD & 0000 & 005 & 0101 & 7FF & 05DC\\
5FD & 1207 & 5FE & 1207 & 5FD & 1207 & 7FF & 05FD & 0040 & 005 & 0101 & &\\
5FE & 2F40 & 5FF & 2F40 & 5FE & 0040 & 7FF & 0040 & 0040 & 001 & 0001 & &\\
5FF & F0FD & 600 & F0FD & 5FF & F0FD & 7FF & 05FF & 0040 & 001 & 0001 & &\\
600 & 1206 & 601 & 1206 & 600 & 1206 & 7FF & 0600 & 0056 & 001 & 0001 & &\\
601 & 2ED1 & 602 & 2ED1 & 5D3 & 00FF & 7FF & FFD1 & 0056 & 001 & 0001 & &\\
602 & 0A00 & 5DC & 0A00 & 7FF & 05DC & 000 & 0602 & 0056 & 001 & 0001 & &\\
5DC & E5D7 & 5DD & E5D7 & 5D7 & 0056 & 000 & 05DC & 0056 & 001 & 0001 & 5D7 & 0056\\
5DD & A5D5 & 5DE & A5D5 & 5D5 & 0001 & 000 & 05DD & 0001 & 001 & 0001 & &\\
5DE & 0700 & 5DF & 0700 & 5DE & 0700 & 000 & 05DE & 0002 & 000 & 0000 & &\\
5DF & E5D5 & 5E0 & E5D5 & 5D5 & 0002 & 000 & 05DF & 0002 & 000 & 0000 & 5D5 & 0002\\
5E0 & 0480 & 5E1 & 0480 & 5E0 & 0480 & 000 & 05E0 & 0001 & 000 & 0000 & &\\
5E1 & F503 & 5E5 & F503 & 5E1 & F503 & 000 & 0003 & 0001 & 000 & 0000 & &\\
5E5 & A5D7 & 5E6 & A5D7 & 5D7 & 0056 & 000 & 05E5 & 0056 & 000 & 0000 & &\\
5E6 & 0680 & 5E7 & 0680 & 5E6 & 0680 & 000 & 05E6 & 5600 & 000 & 0000 & &\\
5E7 & 48EE & 5E8 & 48EE & 562 & 005A & 000 & FFEE & 565A & 000 & 0000 & &\\
5E8 & EAED & 5E9 & EAED & 562 & 565A & 000 & FFED & 565A & 000 & 0000 & 5D6 & 0563,\\
    &      &     &      &     &      &     &      &      &     &      & 562	& 565A\\
5E9 & A5D7 & 5EA & A5D7 & 5D7 & 0056 & 000 & 05E9 & 0056 & 000 & 0000 & &\\
5EA & 7EE7 & 5EB & 7EE7 & 5D2 & 000A & 000 & FFE7 & 0056 & 001 & 0001 & &\\
5EB & F1EE & 5DA & F1EE & 5EB & F1EE & 000 & FFEE & 0056 & 001 & 0001 & &\\
5DA & 0200 & 5DB & 0200 & 5DA & 0200 & 000 & 05DA & 0000 & 005 & 0101 & &\\
5DB & DE21 & 5FD & DE21 & 7FF & 05DC & 7FF & 05FD & 0000 & 005 & 0101 & 7FF & 05DC\\
5FD & 1207 & 5FE & 1207 & 5FD & 1207 & 7FF & 05FD & 0040 & 005 & 0101 & &\\
5FE & 2F40 & 5FF & 2F40 & 5FE & 0040 & 7FF & 0040 & 0040 & 001 & 0001 & &\\
5FF & F0FD & 600 & F0FD & 5FF & F0FD & 7FF & 05FF & 0040 & 001 & 0001 & &\\
600 & 1206 & 601 & 1206 & 600 & 1206 & 7FF & 0600 & 000A & 001 & 0001 & &\\
601 & 2ED1 & 602 & 2ED1 & 5D3 & 00FF & 7FF & FFD1 & 000A & 001 & 0001 & &\\
602 & 0A00 & 5DC & 0A00 & 7FF & 05DC & 000 & 0602 & 000A & 001 & 0001 & &\\
5DC & E5D7 & 5DD & E5D7 & 5D7 & 000A & 000 & 05DC & 000A & 001 & 0001 & 5D7 & 000A\\
5DD & A5D5 & 5DE & A5D5 & 5D5 & 0002 & 000 & 05DD & 0002 & 001 & 0001 & &\\
5DE & 0700 & 5DF & 0700 & 5DE & 0700 & 000 & 05DE & 0003 & 000 & 0000 & &\\
5DF & E5D5 & 5E0 & E5D5 & 5D5 & 0003 & 000 & 05DF & 0003 & 000 & 0000 & 5D5 & 0003\\
5E0 & 0480 & 5E1 & 0480 & 5E0 & 0480 & 000 & 05E0 & 0001 & 003 & 0011 & &\\
5E1 & F503 & 5E2 & F503 & 5E1 & F503 & 000 & 05E1 & 0001 & 003 & 0011 & &\\
5E2 & A5D7 & 5E3 & A5D7 & 5D7 & 000A & 000 & 05E2 & 000A & 001 & 0001 & &\\
5E3 & E8F2 & 5E4 & E8F2 & 563 & 000A & 000 & FFF2 & 000A & 001 & 0001 & 563 & 000A\\
5E4 & CE04 & 5E9 & CE04 & 5E4 & 05E9 & 000 & 0004 & 000A & 001 & 0001 & &\\
5E9 & A5D7 & 5EA & A5D7 & 5D7 & 000A & 000 & 05E9 & 000A & 001 & 0001 & &\\
5EA & 7EE7 & 5EB & 7EE7 & 5D2 & 000A & 000 & FFE7 & 000A & 005 & 0101 & &\\
5EB & F1EE & 5EC & F1EE & 5EB & F1EE & 000 & 05EB & 000A & 005 & 0101 & &\\
5EC & A5D4 & 5ED & A5D4 & 5D4 & 0562 & 000 & 05EC & 0562 & 001 & 0001 & &\\
5ED & E5D6 & 5EE & E5D6 & 5D6 & 0562 & 000 & 05ED & 0562 & 001 & 0001 & 5D6 & 0562\\\hline
\end{tabular}
}
\Table{tracetable-out}{Трассировка программы c момента начала цикла вывода до конца}{
    \begin{tabular}{|*{14}{l|}}
\hline
\multicolumn{2}{|l|}{\begin{tabular}[c]{@{}l@{}}Выполняемая\\команда\end{tabular}} & \multicolumn{9}{l|}{\begin{tabular}[c]{@{}l@{}}Содержимое регистров процессора\\ послевыполнения команды.\end{tabular}} & \multicolumn{2}{l|}{\begin{tabular}[c]{@{}l@{}}Ячейка, содержимое\\которой изменилось\\после выполнения команды\end{tabular}}  \\
\hline
\textbf{Адрес}	& \textbf{Значение}	& \textbf{IP}	& \textbf{CR}	& \textbf{AR}	& \textbf{DR}	&\textbf{SP}	& 	 \textbf{BR}	& \textbf{AC}& \textbf{PS} & \textbf{NZVC}	& \textbf{Адрес}	& \textbf{Значение}\\\hline
5EE & A8E7 & 5EF & A8E7 & 562 & 565A & 000 & FFE7 & 565A & 001 & 0001 & &\\
5EF & DE13 & 603 & DE13 & 7FF & 05F0 & 7FF & 0603 & 565A & 001 & 0001 & 7FF & 05F0\\
603 & 130C & 604 & 130C & 603 & 130C & 7FF & 0603 & 565A & 001 & 0001 & &\\
604 & 120E & 605 & 120E & 604 & 120E & 7FF & 0604 & 565A & 001 & 0001 & &\\
605 & 0480 & 606 & 0480 & 605 & 0480 & 7FF & 0605 & AB2D & 00A & 1010 & &\\
606 & F4FD & 607 & F4FD & 606 & F4FD & 7FF & 0606 & AB2D & 00A & 1010 & &\\
607 & 0A00 & 5F0 & 0A00 & 7FF & 05F0 & 000 & 0607 & AB2D & 00A & 1010 & &\\
5F0 & A8E5 & 5F1 & A8E5 & 562 & 565A & 000 & FFE5 & 565A & 000 & 0000 & &\\
5F1 & 2EE1 & 5F2 & 2EE1 & 5D3 & 00FF & 000 & FFE1 & 005A & 000 & 0000 & &\\
5F2 & 7EDF & 5F3 & 7EDF & 5D2 & 000A & 000 & FFDF & 005A & 001 & 0001 & &\\
5F3 & F008 & 5F4 & F008 & 5F3 & F008 & 000 & 05F3 & 005A & 001 & 0001 & &\\
5F4 & A8E1 & 5F5 & A8E1 & 562 & 565A & 000 & FFE1 & 565A & 001 & 0001 & &\\
5F5 & 0680 & 5F6 & 0680 & 5F5 & 0680 & 000 & 05F5 & 5A56 & 001 & 0001 & &\\
5F6 & DE0C & 603 & DE0C & 7FF & 05F7 & 7FF & 0603 & 5A56 & 001 & 0001 & 7FF & 05F7\\
603 & 130C & 604 & 130C & 603 & 130C & 7FF & 0603 & 5A56 & 001 & 0001 & &\\
604 & 120E & 605 & 120E & 604 & 120E & 7FF & 0604 & 5A56 & 001 & 0001 & &\\
605 & 0480 & 606 & 0480 & 605 & 0480 & 7FF & 0605 & AD2B & 00A & 1010 & &\\
606 & F4FD & 607 & F4FD & 606 & F4FD & 7FF & 0606 & AD2B & 00A & 1010 & &\\
607 & 0A00 & 5F7 & 0A00 & 7FF & 05F7 & 000 & 0607 & AD2B & 00A & 1010 & &\\
5F7 & AADE & 5F8 & AADE & 562 & 565A & 000 & FFDE & 565A & 000 & 0000 & 5D6 & 0563\\
5F8 & 0680 & 5F9 & 0680 & 5F8 & 0680 & 000 & 05F8 & 5A56 & 000 & 0000 & &\\
5F9 & 2ED9 & 5FA & 2ED9 & 5D3 & 00FF & 000 & FFD9 & 0056 & 000 & 0000 & &\\
5FA & 7ED7 & 5FB & 7ED7 & 5D2 & 000A & 000 & FFD7 & 0056 & 001 & 0001 & &\\
5FB & F1F2 & 5EE & F1F2 & 5FB & F1F2 & 000 & FFF2 & 0056 & 001 & 0001 & &\\
5EE & A8E7 & 5EF & A8E7 & 563 & 000A & 000 & FFE7 & 000A & 001 & 0001 & &\\
5EF & DE13 & 603 & DE13 & 7FF & 05F0 & 7FF & 0603 & 000A & 001 & 0001 & 7FF & 05F0\\
603 & 130C & 604 & 130C & 603 & 130C & 7FF & 0603 & 000A & 001 & 0001 & &\\
604 & 120E & 605 & 120E & 604 & 120E & 7FF & 0604 & 000A & 001 & 0001 & &\\
605 & 0480 & 606 & 0480 & 605 & 0480 & 7FF & 0605 & 8005 & 00A & 1010 & &\\
606 & F4FD & 607 & F4FD & 606 & F4FD & 7FF & 0606 & 8005 & 00A & 1010 & &\\
607 & 0A00 & 5F0 & 0A00 & 7FF & 05F0 & 000 & 0607 & 8005 & 00A & 1010 & &\\
5F0 & A8E5 & 5F1 & A8E5 & 563 & 000A & 000 & FFE5 & 000A & 000 & 0000 & &\\
5F1 & 2EE1 & 5F2 & 2EE1 & 5D3 & 00FF & 000 & FFE1 & 000A & 000 & 0000 & &\\
5F2 & 7EDF & 5F3 & 7EDF & 5D2 & 000A & 000 & FFDF & 000A & 005 & 0101 & &\\
5F3 & F008 & 5FC & F008 & 5F3 & F008 & 000 & 0008 & 000A & 005 & 0101 & &\\
5FC & 0100 & 5FD & 0100 & 5FC & 0100 & 000 & 05FC & 000A & 005 & 0101 &     &      \\\hline
\end{tabular}
}
\Section{Программа}
Была составлена программа на языке ассемблера. Она представлена в листингaх \ref{lst:script-setup-and-read} и \ref{lst:script-print-subfunc}.

\refstepcounter{lstlisting}
\begin{figure}[h] %- \usepackage {float} %[h]
    \begin{center}
        \lstinputlisting[]{listings/script-setup-and-read.asm}
    \end{center}
    \captionof{lstlisting}{Код программы на языке ассемблена БЭВМ: параметры программы, переменные и цикл ввода символов}
    \label{lst:script-setup-and-read}
\end{figure}

\refstepcounter{lstlisting}
\begin{figure}[h] %- \usepackage {float} %[h]
    \begin{center}
        \lstinputlisting[]{listings/script-print-subfunc.asm}
    \end{center}
    \captionof{lstlisting}{Код программы на языке ассемблена БЭВМ: цикл вывода символов и подпрограммы для асинхронного ввода и вывода символов}
    \label{lst:script-print-subfunc}
\end{figure}


\Section{Вывод}
Научился красиво называть метки в ассемблере БЭВМ. Научился писать программы c асинхронным вводом-выводом при помощи цикла. Научился работать с УВ текстовым принтером (поработал с хотя бы каким-то принтером).

\newpage
%<<<<<<<<<<<<<<<<<<<<<< КОД РАБОТЫ <<<<<<<<<<<<<<<<<<<<<<<<


%>>>>>>>>>>>>>>>> СПИСОК ЛИТЕРАТУРЫ >>>>>>>>>>>>>>>>>>>>>>>
%\include{biblist}  % Для соответсвия гост, придется доработать. Нужен файл .bib
%<<<<<<<<<<<<<<<<<<<< СПИСОК ЛИТЕРАТУРЫ <<<<<<<<<<<<<<<<<<<


\end{document}
%<<<<<<<<<<<<<<<< ,,,,,,,,,,,,,,,,,,,,,,, <<<<<<<<<<<<<<<<<
%<<<<<<<<<<<<<<<<<<< СОДЕРЖИМОЕ ОТЧЕТА <<<<<<<<<<<<<<<<<<<<
