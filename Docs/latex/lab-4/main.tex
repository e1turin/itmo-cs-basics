%%%%%%%%%%%%%%%%%%%%%%%%%%%%%%%%% LAB-5 %%%%%%%%%%%%%%%%%%%%%%%%%%%%%%%%%%
%>>>>>>>>>>>>>>>>>>>>>>>>>> ПЕРЕМЕННЫЕ >>>>>>>>>>>>>>>>>>>>>>>>>>>>>>>>>>>
%>>>>> Информация о кафедре
%\newcommand{\year}{2021 г.}  % Год устанавливается автоматически
\newcommand{\city}{Санкт-Петербург}  %  Футер, нижний колонтитул на титульном листе
\newcommand{\university}{Национальный исследовательский университет ИТМО}  % первая строка
\newcommand{\department}{Факультет программной инженерии и компьютерной техники}  % Вторая строка
\newcommand{\major}{Направление системного и прикладного программного обеспечения}  % Треьтя строка
%<<<<< Информация о кафедре

%>>>>> Назание работы
\newcommand{\reporttype}{ОТЧЕТ ПО ЛАБОРАТОРНОЙ РАБОТЕ} % тип работы, (главный заголовок титульного листа)
\newcommand{\lab}{Лабораторная работа}          % вид работы
\newcommand{\labnumber}{№ 3}                    % порядковый номер работы
\newcommand{\subject}{Основы профессиональной деятельности}         % учебный предмет
\newcommand{\labtheme}{Исследование работы БЭВМ}            % Тема лабораторной работы
\newcommand{\variant}{№ 1025}                % номер варианта работы

\newcommand{\student}{Тюрин Иван Николаевич}    % определение ФИО студента
\newcommand{\studygroup}{P3110}                 % определение учебной группы 
\newcommand{\teacher}{Клименков С. В.,\\[1mm]     % ФИО лектора
                        Ларочкин Г. И.}          % ФИО практика
%<<<<<<<<<<<<<<<<<<<<<<<<<< ПЕРЕМЕННЫЕ <<<<<<<<<<<<<<<<<<<<<<<<<<<<<<<<<<<


%>>>>>>>>>>>>>>>>>>>>>> ПРЕАМБУЛА >>>>>>>>>>>>>>>>>>>>>>>>>
\include{preamble}
%<<<<<<<<<<<<<<<<<<<<<< ПРЕАМБУЛА <<<<<<<<<<<<<<<<<<<<<<<<<



%%%%%%%%%%%%%%%%%%% СОДЕРЖИМОЕ ОТЧЕТА %%%%%%%%%%%%%%%%%%%%%
%>>>>>>>>>>>>>>> ''''''''''''''''''''''' >>>>>>>>>>>>>>>>>>
\begin{document}


%>>>>>>>>>>>>>>>> ОПРЕДЕЛЕНИЕ НАЗВАНИЙ >>>>>>>>>>>>>>>>>>>>
% Переоформление некоторых стандартных названий
%\renewcommand{\chaptername}{Лабораторная работа}
\renewcommand{\chaptername}{\lab\ \labnumber} % переименование глав
\def\contentsname{Содержание} % переименование оглавления
%<<<<<<<<<<<<<<<< ОПРЕДЕЛЕНИЕ НАЗВАНИЙ <<<<<<<<<<<<<<<<<<<<


%>>>>>>>>>>>>>>>>> ТИТУЛЬНАЯ СТРАНИЦА >>>>>>>>>>>>>>>>>>>>>
\include{titlepage}
%<<<<<<<<<<<<<<<<< ТИТУЛЬНАЯ СТРАНИЦА <<<<<<<<<<<<<<<<<<<<<


%>>>>>>>>>>>>>>>>>>>>> СОДЕРЖАНИЕ >>>>>>>>>>>>>>>>>>>>>>>>>
% Содержание
\tableofcontents
%<<<<<<<<<<<<<<<<<<<<< СОДЕРЖАНИЕ <<<<<<<<<<<<<<<<<<<<<<<<<


%%%%%%%%%%%%%%%%%%%%%%% КОД РАБОТЫ %%%%%%%%%%%%%%%%%%%%%%%%
%>>>>>>>>>>>>>>>>>>>'''''''''''''''''>>>>>>>>>>>>>>>>>>>>>
\newpage
\Chapter{\lab\ \labnumber}{\labtheme}{}

\Section{Задание варианта \variant}
\begin{center}
, , ,
\end{center}
\noindent

\textit{По выданному преподавателем варианту восстановить текст заданного варианта программы, определить предназначение и составить описание программы, определить область представления и область допустимых значений исходных данных и результата, выполнить трассировку программы.}

\textit{Ход работы, содержание отчета и контрольные вопросы описаны в методических указаниях}

\begin{center}
    \includegraphics[width=0.7\paperwidth]{figures/task.png}
\end{center}
\begin{center}
    ' ' '
\end{center}

\newpage
\Section{Описание программы}

\Subsection{Назначение программы и реализуемые ею функция}

Описание программы представлено в таблице 1.1. %\ref{tab:program}.
В результате выполнения программы происходит вычисление суммы: $f(Y+1)+f(X)+1+f(Z)-1$, где $f(n)$ - подпрограмма, принимающая операндом $n$.
\Table{program}{Описание работы программы}{
    %\begin{longtable}{!t}
    \begin{tabular}{|l|l|l|l|}
\hline
\textbf{Адрес} & \textbf{\begin{tabular}[c]{@{}l@{}}Данные/\\ /Команда\end{tabular}} & \textbf{Мнемоника} & \textbf{Описание} \\ \hline
\hex{075} & \hex{0200} & \ttt{cla}         & Очистка AC                      \\ \hline
\hex{076} & \hex{ee19} & \ttt{st VAR}      & Обнуление переменной VAR        \\ \hline
\hex{077} & \hex{ae16} & \ttt{ld Y}        & Загрузка Y в AC                 \\ \hline
\hex{078} & \hex{0700} & \ttt{inc}         & Увеличение AC на 1              \\ \hline
\hex{079} & \hex{0c00} & \ttt{push}        & Положить значение AC на стэк    \\ \hline
\hex{07a} & \hex{d694} & \ttt{call \$FUNC} & Вызов функции FUNC              \\ \hline
\hex{07b} & \hex{0800} & \ttt{pop}         & Загрузка из стэка в AC          \\ \hline
\hex{07c} & \hex{4e13} & \ttt{add VAR}     & Прибавление переменной VAR к AC \\ \hline
\hex{07d} & \hex{ee12} & \ttt{st VAR}      & Сохранение значения в VAR       \\ \hline
\hex{07e} & \hex{ae10} & \ttt{ld X}        & Загрузка переменной X в AC      \\ \hline
\hex{07f} & \hex{0c00} & \ttt{push}        & Положить значение AC на стэк    \\ \hline
\hex{080} & \hex{d694} & \ttt{call \$FUNC} & Вызов функции FUNC              \\ \hline
\hex{081} & \hex{0800} & \ttt{pop}         & Загрузка из стэка в AC          \\ \hline
\hex{082} & \hex{0700} & \ttt{inc}         & Увеличение AC на 1              \\ \hline
\hex{083} & \hex{4e0c} & \ttt{add VAR}     & Прибавление переменной VAR к AC \\ \hline
\hex{084} & \hex{ee0b} & \ttt{st VAR}      & Сохранение значения в VAR       \\ \hline
\hex{085} & \hex{ae07} & \ttt{ld X}        & Загрузка переменной X в AC      \\ \hline
\hex{086} & \hex{0c00} & \ttt{push}        & Положить значение AC на стэк    \\ \hline
\hex{087} & \hex{d694} & \ttt{call \$FUNC} & Вызов функции FUNC              \\ \hline
\hex{088} & \hex{0800} & \ttt{pop}         & Загрузка из стэка в AC          \\ \hline
\hex{089} & \hex{0740} & \ttt{inc}         & Увеличение AC на 1              \\ \hline
\hex{08a} & \hex{4e05} & \ttt{add VAR}     & Прибавление переменной VAR к AC \\ \hline
\hex{08b} & \hex{ee04} & \ttt{st VAR}      & Сохранение значения в VAR       \\ \hline
\hex{08c} & \hex{0100} & \ttt{hlt}         & Останов                         \\ \hline
\hex{08d} & \hex{zzzz} & Z                 & Параметр Z                      \\ \hline
\hex{08e}  & \hex{yyyy} & Y                 & Параметр Y                       \\ \hline
\hex{08f}  & \hex{xxxx} & X                 & Параметр X                      \\ \hline
\hex{090} & \hex{023e} & VAR               & Переменная, хранящая результат                     \\ \hline
\end{tabular}
    %\end{longtable}
}
\Table{program-func}{Описание работы подпрограммы}{
    %\begin{longtable}{!t}
    \begin{tabular}{|l|l|l|l|}
\hline
\textbf{Адрес} & \textbf{\begin{tabular}[c]{@{}l@{}}Данные/\\ /Команда\end{tabular}} & \textbf{Мнемоника} & \textbf{Описание} \\ \hline
\hex{694} & \hex{ac01} & \ttt{ld \&1}        & Загрузка операнда в AC                     \\ \hline
\hex{695} & \hex{f204} & \ttt{bmi LTEQZVL1} & Переход, если операдн отрицательный        \\ \hline
\hex{696} & \hex{f003} & \ttt{beq LTEQZVL1} & Переход, если операнд 0                    \\ \hline
\hex{697} & \hex{7e09} & \ttt{cmp VAL1}     & Сравнение AC с VAL1                        \\ \hline
\hex{698} & \hex{f005} & \ttt{beq LTEQVAL1} & Переход, если операнд равен VAL1           \\ \hline
\hex{699} & \hex{f804} & \ttt{blt LTEQVAL1} & Переход, если операнд меньше VAL1          \\ \hline
\hex{69a} & \hex{4c01} & \ttt{add \&1}       & Увеличить операнд на еще одно его значение, метка LTEQZVL1: \\ \hline
\hex{69b} & \hex{4c01} & \ttt{add \&1}       & Увеличить операнд на еще одно его значение \\ \hline
\hex{69c} & \hex{4e05} & \ttt{add VAL2}     & Увеличить операнд на VAL2                  \\ \hline
\hex{69d} & \hex{ce01} & \ttt{br STOP}      & Переход в конец функции                    \\ \hline
\hex{69e} & \hex{ae02} & \ttt{ld VAL1}      & Приравнять операнд (AC) к VAL1, метка LTEQVAL1:             \\ \hline
\hex{69f} & \hex{ec01} & \ttt{st \&1}        & Сохранить результат функции по (SP+1)      \\ \hline
\hex{6a0} & \hex{0a00} & \ttt{ret}          & Выход из функции                           \\ \hline
\hex{6a1} & \hex{00d0} & \ttt{VAL1}         &                                            \\ \hline
\hex{6a2} & \hex{00b7} & \ttt{VAL2}         &                                            \\ \hline
\end{tabular}
    %\end{longtable}
}
\Subsection{Область представления и допустимых значений}
Значения переменных X, Y, Z ограничены из-за функции $f$ и суммы значений $f$.
Таким образом, крайние нижне и верхнее возможные значения X и Z -- -1881 и 1760, соответственно.
Крайние значения Y -- -1880 и 1761. Они представлены на графике \ref{fig:ODZ}.
Максимальное и минимальнео значение $f$ -- -5461 и 5461.
\Figure{figures/ODZ.png}{\textwidth}{Область допустимых значений параметров X, Y, Z и область значений функции.}{ODZ}

\Subsection{Трассировка программы}

Трассировка программы для параметров X=\hex{00aa}, Y=\hex{00b8}, Z=\hex{ffff} представлена в таблице 1.2.\ref{tab:tracetable}.

\Table{tracetable}{Трассировка программы}{
    \begin{tabular}{|*{11}{l|}}
\hline
Адр	&	МК	&	IP	&	CR	&	AR	&	DR	&	SP	&	BR	&	AC	&	NZVC	&	СчМК	\\\hline
01	&	00A0009004	&	002	&	0000	&	002	&	0006	&	000	&	0002	&	0006	&	0000	&	02	\\
02	&	0104009420	&	003	&	0000	&	002	&	9000	&	000	&	0002	&	0006	&	0000	&	03	\\
03	&	0002009001	&	003	&	9000	&	002	&	9000	&	000	&	0002	&	0006	&	0000	&	04	\\
04	&	8109804002	&	003	&	9000	&	002	&	9000	&	000	&	0002	&	0006	&	0000	&	09	\\
09	&	800C404002	&	003	&	9000	&	002	&	9000	&	000	&	0002	&	0006	&	0000	&	0C	\\
0C	&	8024084002	&	003	&	9000	&	002	&	9000	&	000	&	0002	&	0006	&	0000	&	24	\\
24	&	8026804002	&	003	&	9000	&	002	&	9000	&	000	&	0002	&	0006	&	0000	&	25	\\
25	&	814A404002	&	003	&	9000	&	002	&	9000	&	000	&	0002	&	0006	&	0000	&	26	\\
26	&	0080009001	&	003	&	9000	&	000	&	9000	&	000	&	0002	&	0006	&	0000	&	27	\\
27	&	0100000000	&	003	&	9000	&	000	&	0010	&	000	&	0002	&	0006	&	0000	&	28	\\
28	&	813C804002	&	003	&	9000	&	000	&	0010	&	000	&	0002	&	0006	&	0000	&	3C	\\
3C	&	8143204002	&	003	&	9000	&	000	&	0010	&	000	&	0002	&	0006	&	0000	&	3D	\\
3D	&	81E0104002	&	003	&	9000	&	000	&	0010	&	000	&	0002	&	0006	&	0000	&	E0	\\
E0	&	0001E09611	&	003	&	9000	&	000	&	000A	&	000	&	0002	&	0006	&	0001	&	E1	\\
E1	&	8055101040	&	003	&	9000	&	000	&	000A	&	000	&	0002	&	0006	&	0001	&	55	\\
55	&	0200000000	&	003	&	9000	&	000	&	000A	&	000	&	0002	&	0006	&	0001	&	56	\\
56	&	80C4101040	&	003	&	9000	&	000	&	000A	&	000	&	0002	&	0006	&	0001	&	C4	\\
C4	&	80DE801040	&	003	&	9000	&	000	&	000A	&	000	&	0002	&	0006	&	0001	&	C5	\\
C5	&	8001401040	&	003	&	9000	&	000	&	000A	&	000	&	0002	&	0006	&	0001	&	01	\\\hline
\end{tabular}
}
\Section{Программа}
Была составлена программа на языке ассемблера. Она представлена в листинге \ref{lst:script}.

\refstepcounter{lstlisting}
\begin{figure}[H] %- \usepackage {float} %[h]
    \begin{center}
        \lstinputlisting[]{script-v1-0.asm}
    \end{center}
    \captionof{lstlisting}{Код программы на языке ассемблена БЭВМ}
    \label{lst:script}
\end{figure}

\Section{Эквивалентная программа}
Была составлена эквивалентная программа на языке Python. Она представлена в листинге \ref{lst:funcpy}.

\refstepcounter{lstlisting}
\begin{figure}[H] %- \usepackage {float} %[h]
    \begin{center}
        \lstinputlisting[language=Python]{func.py}
    \end{center}
    \captionof{lstlisting}{Код эквивалентной программы на языке Python}
    \label{lst:funcpy}
\end{figure}

\Section{Вывод}
Научился читать коды БЭВМ, сдерживать ярость, проверять номер варианта. Научился писать программы с ветвлениями и подпрограммами на ассемблере БЭВМ.
% Вывод...
\newpage
%<<<<<<<<<<<<<<<<<<<<<< КОД РАБОТЫ <<<<<<<<<<<<<<<<<<<<<<<<


%>>>>>>>>>>>>>>>> СПИСОК ЛИТЕРАТУРЫ >>>>>>>>>>>>>>>>>>>>>>>
%\include{biblist}  % Для соответсвия гост, придется доработать. Нужен файл .bib
%<<<<<<<<<<<<<<<<<<<< СПИСОК ЛИТЕРАТУРЫ <<<<<<<<<<<<<<<<<<<


\end{document}
%<<<<<<<<<<<<<<<< ,,,,,,,,,,,,,,,,,,,,,,, <<<<<<<<<<<<<<<<<
%<<<<<<<<<<<<<<<<<<< СОДЕРЖИМОЕ ОТЧЕТА <<<<<<<<<<<<<<<<<<<<
